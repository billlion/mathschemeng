\documentclass[11pt,a4paper]{article}

\usepackage[twoside,a4paper,hmarginratio=3:2,vmarginratio=1:1,bmargin=2.54cm]{
geometry}                % See geometry.pdf to learn the layout options. There are lots.
\geometry{a4paper}                   % ... or a4paper or a5paper or ... 
%\geometry{landscape}                % Activate for for rotated page geometry
%\usepackage[parfill]{parskip}    % Activate to begin paragraphs with an empty line rather than an inden

\usepackage{graphicx}
\usepackage{graphics}
\usepackage{amssymb}
\usepackage{amsmath}
\usepackage{epstopdf}
\usepackage{longtable}
\usepackage{lscape}
%\usepackage{savepapr}
\usepackage{fancyhdr}
\usepackage{fancybox}
\usepackage{indentfirst}
\usepackage{ifthen}
\usepackage{comment}
\usepackage{flafter}
%%%\usepackage{reenumi}
\usepackage{bibentry}
%\usepackage{hyperref}

\usepackage[latin1]{inputenc}
\usepackage{amsmath}
\usepackage{amsfonts}
%\usepackage{makeidx}
\usepackage{bm}
\usepackage{multicol}
\usepackage{color}

\begin{document}

\begin{center} 
{\bf Week 1 CHEN10072 \\
Some examples of using MATLAB for checking week 1 problems}
\end{center}

\begin{verbatim}
>>% Q1  This is just a comment!
>> A=[1,2;-2,2] % Note the use of comma and semicolon

A =

     1     2
    -2     2

>> B=[-3,1;-1,-4]

B =

    -3     1
    -1    -4

>> A+ 2*B  % Need a * for multiplication

ans =

    -5     4
    -4    -6

>> A'+B'  %Apostrophe is transpose

ans =

    -2    -3
     3    -2

>> (A+B)'

ans =

    -2    -3
     3    -2

>> A*B

ans =

    -5    -7
     4   -10

>> B*A

ans =

    -5    -4
     7   -10

>> B'*A'

ans =

    -5     4
    -7   -10

>> (A*B)'

ans =

    -5     4
    -7   -10

>> % Q2
>> A=[1;2;-1]

A =

     1
     2
    -1

>> B=[2;-1;0]

B =

     2
    -1
     0

>> A*A'

ans =

     1     2    -1
     2     4    -2
    -1    -2     1

>> A*A'+eye(3)

ans =

     2     2    -1
     2     5    -2
    -1    -2     2

>> det(A*A'+eye(3))

ans =

     7

>> 
\end{verbatim}



\hfill {\tiny Last modified  \today}

\end{document}
