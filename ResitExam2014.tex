\documentclass[12pt]{article}
%\usepackage{mathexam,amssymb,theorem}
\usepackage{CEASexam,amsfonts}
\usepackage {amsmath,amssymb}


\coursename{Engineering Mathematics 2}
\examinationdate{Resit  Examination}
\examinationtime{ xxx -- yyy }

\coursecode{CHEN10072}
\duration{One and a Half  Hours}
\rubric{
To be Provided by Examinations Office: Mathematical Formula Tables. \\
Answer {\bf all} parts of  {\bf Section A}
(40 marks in total)
        and\\
        {\bf one} of the two questions in {\bf Section B}
(10 marks)\\
EXAMINATION PAPER NOT TO BE REMOVED FROM EXAMINATION ROOM.
}


\calculators{
Electronic calculators may be used, provided that
they CANNOT store text. }
% The total number of marks on the paper is 50.  This represents 80%
% of the MATH36041 mark and 55% of the MATH66041 mark


%%%%%%%%%%%%%%%%%%%%%%%%%%%%%%%%%%
% Make uppercase Greek characters italic.
% Copied from latex.ltx and changed second digit from 0 (roman font)
% to 1 (math italic).
\mathchardef\Gamma="7100
\mathchardef\Delta="7101
\mathchardef\Theta="7102
\mathchardef\Lambda="7103
\mathchardef\Xi="7104
\mathchardef\Pi="7105
\mathchardef\Sigma="7106
\mathchardef\Upsilon="7107
\mathchardef\Phi="7108
\mathchardef\Psi="7109
\mathchardef\Omega="710A
%%%%%%%%%%%%%%%%%%%%%%%%%%%%%%%%%
% Define a few macros specific to this exam and this particular typist.
\def\ffrac#1#2{\leavevmode\kern.1em
\raise.5ex\hbox{\the\scriptfont0 #1}\kern-.1em
/\kern-.15em\lower.25ex\hbox{\the\scriptfont0 #2}}
\def\half{\frac{1}{2}}
\def\hhalf{\ffrac{1}{2}}
\def\bA{\mathbf{A}}
\def\bB{\mathbf{B}}
\def\bp{\mathbf{p}}
\def\bC{\mathbf{C}}
\def\bD{\mathbf{D}} 
\def\bI{\mathbf{I}}
\def\bP{\mathbf{P}}
\def\bX{\mathbf{X}}
\def\tough{$\!\!\!{}^\star\>$}
\newcommand{\R}{{\mathbb{R}}}
\newcommand{\dif}{\mathsf{d}}
\newcommand{\EndSec}[1]{\begin{center}\bf END OF SECTION #1\end{center}}

\begin{document}
\maketitle

\null\bigskip
{\bf General Instruction.} All numerical working must be shown.


\section
\sectionheader{Answer \jazzy{all} questions}


\question
 Let
\[ \bA \,=\, \left( \begin{array}{rr} 1 & 2 \\ 3 & 0 \end{array} \right) \quad
\hbox{and} \quad
\bB \,=\, \left( \begin{array}{rr} -1 & 0 \\ 1 & -1 \end{array} \right).\]

Compute the following matrices:
	\begin{enumerate}
	\item $2\bA +\bB$.  Is it true that $ 2\bA + \bB= \bB + 2\bA$?
	
	\item $(\bB^T)^{-1}$     Is it true that $ (\bB^T)^{-1} =(\bB^{-1})^{T}$?

	\marks{6}
	\end{enumerate}
	
%\Endquestion

\question
Given 
\[ \bA = \left( \begin{array}{rr} a & b \\ c & d \end{array} \right),\quad \bB = \left( \begin{array}{rr} d & -b \\ -c & a \end{array} \right)\]
show by performing the multiplication that 
\[ \bA \bB = \left( \begin{array}{rr} ad -bc & 0 \\ 0 & ad-bc \end{array} \right) \]
Assuming $\mathrm{det}{\bA}=0$ but $\bA\ne \mathbf{0}$ find a non-zero vector $\bp$ such that $\bA \bp= \mathbf{0}$. 

 \marks{4}
 
%\Endquestion


\question

Given
\[ \bA=   \left ( \begin{array}{rrr} 1 & 2 & 2 \\ 5 & 0 & 1 \\ 1 & 0 & -1 \end{array} \right),\quad \bB=   \left ( \begin{array}{rrr} 1 & 0 & -1\\ 5 & 0 & 1 \\  1 & 2 & 2  \end{array} \right) \]
\begin{enumerate}
\item Evaluate $\mathrm{det}(\bA)$ and $\mathrm{det}(\bB)$
\item Explain why the two determinants are related.
\end{enumerate}


\marks{4}

%\Endquestion

\newpage
\question
Given
\vspace{-0.5cm}
\begin{eqnarray*}
x\,+\,y\,+\,z&=&\,6\\
x\,-\,y\,+\,3z&=&4\\
-3x\,+ \,y -\,z&=&\,-8 .
\end{eqnarray*}
            \begin{enumerate}
            \item Write the system of  equations in the form of $\bA \mathbf{x}=\mathbf{b}$.
            \item Determine whether or not {\bf A} is singular.
            \item Solve for $x,y,z$ using Gaussian elimination {\bf without} pivoting.\marks{7}
    \end{enumerate}
	
	%\Endquestion

\question
Solve analytically the following differential equation problem: find $y(x)$ such that
$$ \frac{\dif y}{\dif x}\,=\, -x(1\,+\,y^2)   ; \qquad y=1 \mbox{ when } x=0. $$
\noindent
 [Hint:   use the fact that
$\displaystyle \int {1 \over {a^2+x^2}} \, \dif x=  \frac{1}{a}\tan^{-1} {x \over a},\ a\ne 0$.]
\marks{4}

%\Endquestion

\question
Solve the following problem by finding an  integrating factor:  find $y(x)$ such that
$$  \frac{\dif y}{\dif x}\,+\, \frac{y}{x} = x^2   ; \qquad y=0\>\hbox{ when }\> x=1. $$
\marks{5}

%\Endquestion

\question
Compute a numerical solution to the differential equation problem
$$ \frac{\dif y}{\dif t} =  -2y +t ; \qquad y=1\>\hbox{ when }\> t=0, $$
by taking three steps of the forward Euler method with
step size $h=0.05$.
\marks{4}

%\Endquestion

\question
Write down  Newton's iteration  for finding a root of
a function $f(x)$. Demonstrate this method applied to finding the positive square root of $2$ correct to four significant figures using only addition, subtraction, multiplication and division.
\marks{6}

%\Endquestion

\EndSec{A}

%%%%%%%%%%%%%%%%%%%%%%%%%%%%% SECTION B
\section
\sectionheader{Answer \jazzy{one} of the two questions}


\question
Use Gaussian Elimination to compute  the inverse of  the following matrix
$$ \bA = \left( \begin{array}{rrr} 1 & 2  &1\\ 1 & 1 & 1\\0 & 0 & 4 \end{array} \right). $$
\null\vspace{-0.6cm}
 \marks{10}
 
%\Endquestion
 

\bigskip
\question
We want to solve the following  linear  {\it  second order} differential equation
problem: find  $y(t)$ such that
$$
{\dif^2 y\over \dif t^2} + 9\,y =0 ; \quad y(0)=0; \>\> {\dif y\over \dif t} (0)=1.
$$

\begin{enumerate}
\item
Show that this problem can be rewritten  as a system of
first order differential equations, and find the associated initial conditions.
\null\vspace{-0.2cm} \marks{3}

\item
 Take a  step size  of $h=0.1$  and
compute the first two steps of the  {\it implicit} (Backward) Euler  method
 so as to approximate $y(0.2)$.  Give a reason why the {\it implicit} Euler
 method might be preferable
 to the  {\it forward} Euler method when solving this problem.
 \null\vspace{-0.2cm} \marks{7}

\end{enumerate}

%\Endquestion
\EndSec{B}
\finished

\end{document}
