\documentclass[11pt,a4paper]{article}

\usepackage[twoside,a4paper,hmarginratio=3:2,vmarginratio=1:1,bmargin=2.54cm]{
geometry}                % See geometry.pdf to learn the layout options. There are lots.
\geometry{a4paper}                   % ... or a4paper or a5paper or ... 
%\geometry{landscape}                % Activate for for rotated page geometry
%\usepackage[parfill]{parskip}    % Activate to begin paragraphs with an empty line rather than an inden

\usepackage{graphicx}
\usepackage{graphics}
\usepackage{amssymb}
\usepackage{amsmath}
\usepackage{epstopdf}
\usepackage{longtable}
\usepackage{lscape}
%\usepackage{savepapr}
\usepackage{fancyhdr}
\usepackage{fancybox}
\usepackage{indentfirst}
\usepackage{ifthen}
\usepackage{comment}
\usepackage{flafter}
%%%\usepackage{reenumi}
\usepackage{bibentry}
%\usepackage{hyperref}

\usepackage[latin1]{inputenc}
\usepackage{amsmath}
\usepackage{amsfonts}
%\usepackage{makeidx}
\usepackage{bm}
\usepackage{multicol}
\usepackage{color}

\newcommand{\dif}{\mathsf{d}}

%%---------------------------------------------------------------  DJS Definitions 
\def\ffrac#1#2{\leavevmode\kern.1em
\raise.5ex\hbox{\the\scriptfont0 #1}\kern-.1em
/\kern-.15em\lower.25ex\hbox{\the\scriptfont0 #2}}
\def\half{\frac{1}{2}}
\def\hhalf{\ffrac{1}{2}}
\def\bA{\mathbf{A}}
\def\bB{\mathbf{B}}
\def\bb{\mathbf{b}}
\def\bC{\mathbf{C}}
\def\bD{\mathbf{D}}
\def\bI{\mathbf{I}}
\def\bP{\mathbf{P}}
\def\bX{\mathbf{X}}
\def\bx{\mathbf{x}}
\def\tough{$\!\!\!{}^\star\>$}
\newcommand{\R}{{\mathbb{R}}}


%% Change this boolean to true to compile solutions.
\newboolean{mynotes}
\setboolean{mynotes}{true}

%%---------------------------------------------------------------------------------------
\begin{document}

\begin{center} 
{\bf Maths Problems for CHEN10072 \\
Bill Lionheart\\
 January 2014}
\end{center}


%%---------------------------------------------------------------------------------------
\section*{Week 16}



\begin{enumerate}

\item
Write the following system of equations in the form $\bA\bx=\bb$
\begin{eqnarray*}
x\,+\,3y\,+\,2z&=&1\\
\,-\,2y\,+\,z&=&2\\
\,\,\,z&=&1
\end{eqnarray*}
and solve the system by inverting the matrix $\bA$

\ifthenelse{\boolean{mynotes}}{
\noindent\hrulefill

{\bf Solution}

\begin{equation*}\bA^{-1}=\left(
\begin{array}{rrr}
 1 & \frac{3}{2} &
   -\frac{7}{2} \\
 0 & -\frac{1}{2} &
   \frac{1}{2} \\
 0 & 0 & 1
\end{array}
\right),\qquad \bx = \bA^{-1}\bb =\left(
\begin{array}{r}
 \frac{1}{2} \\
 -\frac{1}{2} \\
 1
\end{array}
\right)
\end{equation*}
\noindent\hrulefill
}{}%

\item Solve the system in Q1 by back substitution.

\ifthenelse{\boolean{mynotes}}{
\noindent\hrulefill

{\bf Solution}

$z=1, y=-1/2,x=1/2$

\noindent\hrulefill
}{}%

\item Write the following system of equations in the form $\bA\bx=\bb$
\begin{eqnarray*}
x\,+\,2y\,+\,z&=&1\\
3x\,+\,2y\,+\,z&=&2\\
x\,\,\,\,\,\,\,-\,3z&=&1
\end{eqnarray*}
and solve the system by inverting the matrix $\bA$


\ifthenelse{\boolean{mynotes}}{
\noindent\hrulefill

{\bf Solution}

\begin{equation*}\bA =\left(
\begin{array}{ccc}
 1 & 2 & 1 \\
 3 & 2 & 1 \\
 1 & 0 & -3
\end{array}
\right),\quad \bb=\left(
\begin{array}{r}
 1 \\
 2 \\
 1
\end{array}
\right),
\quad\bA^{-1}=\left(
\begin{array}{rrr}
 -\frac{1}{2} & \frac{1}{2} &
   0 \\
 \frac{5}{6} & -\frac{1}{3} &
   \frac{1}{6} \\
 -\frac{1}{6} & \frac{1}{6} &
   -\frac{1}{3}
\end{array}
\right),\quad \bA^{-1}\bb = \left(
\begin{array}{r}
 \frac{1}{2} \\
 \frac{1}{3} \\
 -\frac{1}{6}
\end{array}
\right)
\end{equation*}

\noindent\hrulefill
}{}%
\item Solve the system in Q3 by Gaussian elimination.


\ifthenelse{\boolean{mynotes}}{
\noindent\hrulefill

{\bf Solution}
$x=1/2,y=1/3,z=-1/6$
}{}%
\item\label{q1206}
Potassium chromate is recovered from a methanol/water solution by evaporation, crystallisation
and filtration. The total mass flows, in units of kg~h$^{-1}$, is $E$ for evaporation, $F$ for filtration and $C$
for crystallisation. The material balance equations for potassium chromate, water and methanol are
\begin{eqnarray*}
1125&=&1.06C\,+\,0.3F\\
2250&=&0.75E\,+\,0.08C\,+\,0.4F\\
1125&=&0.25E\,+\,0.06C\,+\,0.3F .
\end{eqnarray*}
\begin{enumerate}
 \item Write the system of  equations in the form 
       $ \bA \mathbf{x}=\mathbf{b}, $
        where {\bf A} is a $3\times 3$ matrix, {\bf x} is a column vector of the unknown quantities and {\bf b}
is a column vector with known coefficients.
        \item Apply  Gaussian elimination to show that $E$, $C$ and $F$ satisfy the upper triangular system
       $$\left ( \begin{array}{rrrr}
         1.06 & 0 & 0.3  \\
        0 &  0.75& 0.37736 \\
        0  & 0 & 0.15723
         \end{array} \right )
   \left ( \begin{array}{c}
        C \\ E \\ F
            \end{array} \right) =
   \left ( \begin{array}{r}
        1125.00 \\ 2165.10 \\ 339.62  
            \end{array} \right) .$$
          \item  Hence calculate $E$, $C$ and $F$ using  back-substitution.
           \item  Calculate the determinant  of the matrix $\bA$. 
           \emph{Hint: Adding multiples of one row to another does not change the determinant of a matrix.}

       \end{enumerate}
       
 

\ifthenelse{\boolean{mynotes}}{
\noindent\hrulefill

{\bf Solution}  
For example    
\begin{enumerate}       
\item \begin{equation*}\bA=\left(
\begin{array}{ccc}
 1.06 & 0 & 0.3 \\
 0.08 & 0.75   & 0.4 \\
 0.06 & 0.25 &   0.3
\end{array}
\right),\quad \bb= \left(
\begin{array}{r}
 1125.0 \\
 2250 \\
 1125
\end{array}
\right),\quad \bx= \left(
\begin{array}{r}
 C \\
 E \\
 F
\end{array}
\right)
\end{equation*}
\item $F=339.62/0.15723=2160.02$ , $E=(0.37736\times F  -  2165.10)/0.75=-1799.99$,
$C=(1125.0 - 0.3\times F)/1.06=449.994$, which we should probably round to two decimal places $C=449.99,E=-1799.99,F=2160.02$.  Note there is some loss of precision here
\item $1.06 \times 0.75 \times 0.15723=0.125$ so determinant is   $0.125$ . 
\end{enumerate}

\noindent\hrulefill
  
}{}%

\item\label{qdjsx1} 
It is known  that the number of floating point operations (flop) required to solve
an $n\times n$ system of linear equations using Gaussian elimination
is approximately $2 n^3 /3$ for large values of $n$. Estimate the 
size of the biggest system that 
can be solved for \pounds 10,000 if it costs \pounds 1,000 an hour to
use the IBM Roadrunner supercomputer which has a calculation rate of $10^{15}$ flop/sec. 
(This is called a {\it petaflop}. Conventional laptop computers run  at  a  {\it gigaflop},  that is $10^{9}$ flop/sec.)

\ifthenelse{\boolean{mynotes}}{
\noindent\hrulefill

{\bf Solution}

Number of flops is $2n^3/3$, so the cost in pounds for $n$ is $\frac{2n^3}{3}\times 10^{-15}\times\frac{1}{60^2}\times 1000$ which is $\frac{n^3}{5400000000000000}$. So the cost is \pounds 10,000 when $n= 3000000 \times 2^{1/3}$ or approximately $3.8\times 10^6$

\noindent\hrulefill

}{}%

\item\label{q1207} Consider the linear equation system:
\begin{eqnarray*}
x\,+\,y\,+\,z&=&3\\
x\,-\,y\,+\,2z&=&2\\
2y\,-\,z&=&5
\end{eqnarray*}
            \begin{enumerate}
            \item Write the system of  equations in the form of $\bA \mathbf{x}=\mathbf{b}$.
            \item  Solve for $x,y,z$ using Gaussian elimination.
             Can you explain this result? 
            \end{enumerate}

\ifthenelse{\boolean{mynotes}}{
\noindent\hrulefill

{\bf Solution}


\begin{enumerate}
\item $$\bA=\left(
\begin{array}{ccc}
 1 & 1 & 1 \\
 1 & -1 & 2 \\
 0 & 2 & -1
\end{array}
\right)$$

\item $$\left(
\begin{array}{rrr|r}
 1 & 0 & \frac{3}{2} & 0 \\
 0 & 1 & -\frac{1}{2} & 0 \\
 0 & 0 & 0 & 1
\end{array}
\right)$$ The determinant of $\bA$ is zero so we might expect not to be able to solve the system. The last equation in the row-reduced form says $0=1$ so there can be no solution. This means that the original system is inconsistent and there are no solutions.
\end{enumerate}

\noindent\hrulefill


}{}%


\item\label{qdjsx2} \tough
Consider the coefficient matrix
 $$ \bA = \left ( \begin{array}{rrrr}
         2 & 4 & -2  \\
        2 &  8& 4 \\
        1  & -2 & -1
         \end{array} \right )$$
   Using Gaussian elimination,  solve the three distinct systems 
  $$ \bA  \mathbf{x}_1=\mathbf{e}_1,\qquad  \bA  \mathbf{x}_2=\mathbf{e}_2, 
   \qquad\bA  \mathbf{x}_3=\mathbf{e}_3,\qquad\qquad\qquad\qquad\qquad $$
  where $\mathbf{e}_1=  \left[ \begin{array}{c} 1 \\ 0\\ 0 \end{array} \right]$, 
   $\mathbf{e}_2=  \left[ \begin{array}{c} 0 \\ 1\\ 0 \end{array} \right]$
    $\mathbf{e}_3=  \left[ \begin{array}{c} 0 \\ 0\\ 1 \end{array} \right]$, \\[2ex]
  so as to  compute the columns of the matrix   $\bX$ satisfying $\bA \bX = \bI$.  (Note that this is
  clever way to compute the inverse matrix $\bX= \bA^{-1}$.)  


\ifthenelse{\boolean{mynotes}}{
\noindent\hrulefill

{\bf Solution}

The $\bx_i$ should be the columns of 
$$\bA^{-1}=\left(
\begin{array}{rrr}
 0 & \frac{1}{6} & \frac{2}{3}
   \\
 \frac{1}{8} & 0 &
   -\frac{1}{4} \\
 -\frac{1}{4} & \frac{1}{6} &
   \frac{1}{6}
\end{array}
\right)$$

\noindent\hrulefill

}{}%



\end{enumerate}
\hfill {\tiny Last modified  \today}
\vfill\eject
%%---------------------------------------------------------------------------------------

\end{document}
