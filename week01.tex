\documentclass[11pt,a4paper]{article}

\usepackage[twoside,a4paper,hmarginratio=3:2,vmarginratio=1:1,bmargin=2.54cm]{
geometry}                % See geometry.pdf to learn the layout options. There are lots.
\geometry{a4paper}                   % ... or a4paper or a5paper or ... 
%\geometry{landscape}                % Activate for for rotated page geometry
%\usepackage[parfill]{parskip}    % Activate to begin paragraphs with an empty line rather than an inden

\usepackage{graphicx}
\usepackage{graphics}
\usepackage{amssymb}
\usepackage{amsmath}
\usepackage{epstopdf}
\usepackage{longtable}
\usepackage{lscape}
%\usepackage{savepapr}
\usepackage{fancyhdr}
\usepackage{fancybox}
\usepackage{indentfirst}
\usepackage{ifthen}
\usepackage{comment}
\usepackage{flafter}
%%%\usepackage{reenumi}
\usepackage{bibentry}
%\usepackage{hyperref}

\usepackage[latin1]{inputenc}
\usepackage{amsmath}
\usepackage{amsfonts}
%\usepackage{makeidx}
\usepackage{bm}
\usepackage{multicol}
\usepackage{color}




\newcommand{\dif}{\mathsf{d}}

%%---------------------------------------------------------------  DJS Definitions 
\def\ffrac#1#2{\leavevmode\kern.1em
\raise.5ex\hbox{\the\scriptfont0 #1}\kern-.1em
/\kern-.15em\lower.25ex\hbox{\the\scriptfont0 #2}}
\def\half{\frac{1}{2}}
\def\hhalf{\ffrac{1}{2}}
\def\bA{\mathbf{A}}
\def\bB{\mathbf{B}}
\def\bC{\mathbf{C}}
\def\bD{\mathbf{D}}
\def\bI{\mathbf{I}}
\def\bP{\mathbf{P}}
\def\bX{\mathbf{X}}
\def\tough{$\!\!\!{}^\star\>$}
\newcommand{\R}{{\mathbb{R}}}


%% Change this boolean to true to compile solutions.
\newboolean{mynotes}
\setboolean{mynotes}{true}

%\setboolean{mynotes}{true}

%%---------------------------------------------------------------------------------------
\begin{document}

\begin{center} 
{\bf Maths Problems for  CHEN10072 \\
\ifthenelse{\boolean{mynotes}}{(With solutions) \\}

 Bill Lionheart
}
\end{center}
\hrule
\smallskip



\section*{Week 1}

\begin{enumerate}

\item\label{q1167} Let
\[ \bA \,=\, \left( \begin{array}{rr} 1 &2 \\ -2 &2 \end{array} \right) \quad
\hbox{and} \quad
\bB \,=\, \left( \begin{array}{rr} -3 &1 \\ -1 &-4 \end{array} \right).\]

Compute the following matrices:
	\begin{enumerate}
	\item $\bA+ 2\bB$.  Is it true that $ \bA + 2\bB= 2\bB + \bA$?
	\item $\bA^T+  \bB^T$. Is it true that $( \bA + \bB)^T = \bA^T + \bB^T$?
	\item $\bA \bB$ and $\bB \bA$.    Is it true that $ \bA \bB= \bB \bA$?
	\item $\bB^T \bA^T$.  Is it true that $(\bA \bB)^T= \bB^T \bA^T$?
	\end{enumerate}

\ifthenelse{\boolean{mynotes}}{
\noindent\hrulefill

{\bf Solution}

\begin{enumerate}
	\item\[\left(
\begin{array}{cc}
 -5 & 4 \\
 -4 & -6
\end{array}
\right).\]
Yes
\item 
\[\left(
\begin{array}{cc}
 -2 & -3 \\
 3 & -2
\end{array}
\right)\]. Yes
\item 
\[ \bA \bB = \left(
\begin{array}{cc}
 -5 & -7 \\
 4 & -10
\end{array}
\right), \quad \bB\bA= \left(
\begin{array}{cc}
 -5 & -4 \\
 7 & -10
\end{array}
\right),\quad \bA \bB \ne \bB \bA \]
\item \[\bB^T \bA^T=\left(
\begin{array}{cc}
 -5 & 4 \\
 -7 & -10
\end{array}
\right)= (\bA \bB)^T\]
\end{enumerate}


\noindent\hrulefill

}{}%



\item\label{qwrblm1}
Let 
\[ \bA =  \left( \begin{array}{r} 1 \\ 2 \\ -1 \end{array} \right) \quad
\hbox{and} \quad
\bB =  \left( \begin{array}{r} 2 \\ -1 \\ 0 \end{array} \right)
\] 
\begin{enumerate}
	\item Calculate $\bA^T \bA$.
	\item Calculate $\bA^T \bB$.
	\item What do the above mean if we think of the matrices as vectors?
	\item Calculate $\bA \bA^T$ and $\bA \bA^T \bB$.
	\item Calculate $\bA \bA^T + \bI$ where $\bI$ is the $3\times 3 $ identity matrix.
	\item\tough Calculate $\det\left(\bA \bA^T + \bI\right)$. How do we know $\det\left(\bA \bA^T\right)=0$ without working out the determinant?
	\end{enumerate}

\ifthenelse{\boolean{mynotes}}{
\noindent\hrulefill

{\bf Solution}

\begin{enumerate}
	\item 6
	\item $0$, or $(0)$
	\item The squared length $||\bA||^2=6$ and the dot product $\bA \cdot \bB=0$ so the two vector are orthogonal. 
	\item \[  \bA \bA^T=\left(
\begin{array}{ccc}
 1 & 2 & -1 \\
 2 & 4 & -2 \\
 -1 & -2 & 1
\end{array}
\right) \]

\item \[ \bA \bA^T + \bI= \left(
\begin{array}{ccc}
 2 & 2 & -1 \\
 2 & 5 & -2 \\
 -1 & -2 & 2
\end{array}
\right) \]
\item \[ \det( \bA \bA^T + \bI)=7 \] as $\bA \bA^T \bB=0$ we see that there is a linear relationship between the columns of $\bA \bA^T$.

\end{enumerate}
\noindent\hrulefill	

}{}%	
\item\label{qdjsp1}
A {\it permutation} matrix  $\bP_{ij}$ is generated by (repeatedly) interchanging rows (or columns) 
of an {\it identity} matrix $\bI$, e.g.
$$  \bP_{12} =  \left ( \begin{array}{ccc}  0 & 1 & 0\\  1 & 0 & 0\\  0 & 0 & 1 \end{array} \right ),  \quad
  \bP_{13} =  \left ( \begin{array}{ccc}  0 & 0 & 1\\  0 & 1 & 0\\  1 & 0 & 0 \end{array} \right )  $$
Permutation matrices always have exactly one $1$ in each row and column with the other elements zero.

Let $\bA$ be the matrix
$$  \bA =  \left ( \begin{array}{ccc}  11 & 12 & 13\\  21 & 22 & 23 \\  31 & 32 & 33 \end{array} \right ).$$
\begin{enumerate}
\item Compute the products $\bP_{12} \bA$,  $\bP_{13} \bA$,   $\bA\bP_{12}$ and  $\bA\bP_{13}$.
What do you observe?
\item  Show that  $\bP_{12} \bP_{12}^T = \bI$.  (This means that $\bP_{12}^T$ is the inverse of   $\bP_{12}$---this
always true for permutation matrices.)
\item Show that the matrix $\bP= \bP_{12} \bP_{13} $ is also a permutation matrix. 
\item Compute  the determinant of $\bP_{12}$ and that of $\bP_{13}$.
\item\tough Use the fact that  $\hbox{det}(\bA \bB) =  \hbox{det}(\bA) \cdot \hbox{det}(\bB)$ to
prove that the determinant of a permutation matrix is $\pm 1$. \emph{Hint: 
use the fact that transposing a matrix does not change its determinant.}


\end{enumerate}
\ifthenelse{\boolean{mynotes}}{

{\bf Solution}

\noindent\hrulefill
\begin{enumerate}
\item
$\bP_{12} \bA= \left(
\begin{array}{ccc}
 21 & 22 & 23 \\
 11 & 12 & 13 \\
 31 & 32 & 33
\end{array}
\right)$,  $\bP_{13} \bA=\left(
\begin{array}{ccc}
 31 & 32 & 33 \\
 12 & 22 & 23 \\
 11 & 12 & 13
\end{array}
\right)$,   $\bA\bP_{12}=\left(
\begin{array}{ccc}
 12 & 11 & 13 \\
 22 & 12 & 23 \\
 32 & 31 & 33
\end{array}
\right)$ and  $\bA\bP_{13}=\left(
\begin{array}{ccc}
 13 & 12 & 11 \\
 23 & 22 & 12 \\
 33 & 32 & 31
\end{array}
\right)$.
\item (Just do it)
\item $\bP=\left(
\begin{array}{ccc}
 0 & 1 & 0 \\
 0 & 0 & 1 \\
 1 & 0 & 0
\end{array}
\right)$, this is a permutation matrix as it has exactly one one in each row and column.
\item Both $-1$.
\item Let $\bP$ be any permutation matrix then $\bP \bP^T=\bI$ so $\det(\bP \bP^T)=\left( \det(\bP)\right)^2= \det(\bI) =1$ so $\det(\bP)=\pm 1$.

\end{enumerate}

\noindent\hrulefill

}{}%


\end{enumerate}
\hfill {\tiny Last modified  \today}
\vfill\eject



\end{document}
