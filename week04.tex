\documentclass[11pt,a4paper]{article}

\usepackage[twoside,a4paper,hmarginratio=3:2,vmarginratio=1:1,bmargin=2.54cm]{
geometry}                % See geometry.pdf to learn the layout options. There are lots.
\geometry{a4paper}                   % ... or a4paper or a5paper or ... 
%\geometry{landscape}                % Activate for for rotated page geometry
%\usepackage[parfill]{parskip}    % Activate to begin paragraphs with an empty line rather than an inden

\usepackage{graphicx}
\usepackage{graphics}
\usepackage{amssymb}
\usepackage{amsmath}
\usepackage{epstopdf}
\usepackage{longtable}
\usepackage{lscape}
%\usepackage{savepapr}
\usepackage{fancyhdr}
\usepackage{fancybox}
\usepackage{indentfirst}
\usepackage{ifthen}
\usepackage{comment}
\usepackage{flafter}
%%%\usepackage{reenumi}
\usepackage{bibentry}
%\usepackage{hyperref}

\usepackage[latin1]{inputenc}
\usepackage{amsmath}
\usepackage{amsfonts}
%\usepackage{makeidx}
\usepackage{bm}
\usepackage{multicol}
\usepackage{color}

\newcommand{\dif}{\mathsf{d}}

%%---------------------------------------------------------------  DJS Definitions 
\def\ffrac#1#2{\leavevmode\kern.1em
\raise.5ex\hbox{\the\scriptfont0 #1}\kern-.1em
/\kern-.15em\lower.25ex\hbox{\the\scriptfont0 #2}}
\def\half{\frac{1}{2}}
\def\hhalf{\ffrac{1}{2}}
\def\bA{\mathbf{A}}
\def\bB{\mathbf{B}}
\def\bC{\mathbf{C}}
\def\bD{\mathbf{D}}
\def\bI{\mathbf{I}}
\def\bP{\mathbf{P}}
\def\bX{\mathbf{X}}
\def\tough{$\!\!\!{}^\star\>$}
\newcommand{\R}{{\mathbb{R}}}


%% Change this boolean to true to compile solutions.
\newboolean{mynotes}
\setboolean{mynotes}{true}

%\setboolean{mynotes}{false}

%%---------------------------------------------------------------------------------------
\begin{document}

\begin{center} 
{\bf Maths Problems for  CHEN10072 \\
\ifthenelse{\boolean{mynotes}}{(With solutions) \\}

 Bill Lionheart
}
\end{center}
\hrule
\smallskip


%%---------------------------------------------------------------------------------------
\section*{Week 4}

\begin{enumerate}


\item\label{qdjsx3}
A paint company is trying to use up excess quantities of four shades
of green paint by mixing them to form a more popular shade. One gallon
of the new paint will be made up of $x_1$ gallons of paint~1, $x_2$
gallons of paint~2 etc. Each of the paints is made up of four
pigments. If each number repesents a percentage, 
the mixture giving the more popular shade is the solution of 
the system of equations
$$\left ( \begin{array}{rrrr}
         0 & 80 & 10 & 10 \\
        80 &  0 & 30 & 10 \\
        16 & 20 & 60 & 72 \\
         4 &  0 &  0 &  8 \end{array} \right )
   \left ( \begin{array}{c}
        x_1 \\ x_2 \\ x_3 \\ x_4 
            \end{array} \right) =
   \left ( \begin{array}{r}
        27 \\ 40 \\ 31 \\ 2 
            \end{array} \right) .$$
Find the optimal mixture by solving the system using Gaussian elimination 
with {\it partial pivoting}. (This means that you interchange rows at every step
to put the biggest number in modulus  on the diagonal. In this example,
 at the first stage of the elimination you would interchange row one with row two.)


You can check your results   in MATLAB (see also the following  exercise) by typing the command sequence
\begin{verbatim}
>> A=[0,80,10,10;80,0,30,10;16,20,60,72;4,0,0,8],  [L,U,P] = lu(A),
>> b=[27;40;31;2], x=A\b, 
\end{verbatim}

\ifthenelse{\boolean{mynotes}}{
\noindent\hrulefill
%\begin{verbatim}
%>>A=[0,80,10,10;80,0,30,10;16,20,60,72;4,0,0,8],  [L,U,P] = lu(A),
%A =
%     0    80    10    10
%    80     0    30    10
%    16    20    60    72
%     4     0     0     8
%L =
%    1.0000         0         0         0
%         0    1.0000         0         0
%    0.2000    0.2500    1.0000         0
%    0.0500         0   -0.0291    1.0000
%U =
%   80.0000         0   30.0000   10.0000
%         0   80.0000   10.0000   10.0000
%         0         0   51.5000   67.5000
%         0         0         0    9.4660
%P =
%     0     1     0     0
%     1     0     0     0
%     0     0     1     0
%     0     0     0     1
%>> b=[27;40;31;2], x=A\b, 
%b =
%    27
%    40
%    31
%     2
%x =
%    0.4000
%    0.3000
%    0.2500
%    0.0500
%\end{verbatim}
\noindent\hrulefill
}{}

\item\label{qdjsx4}
Given a matrix
$$\bA= \left ( \begin{array}{rrr}
         4.00 & -2.00 & 3.00\\
        -2.00 &  7.25 & -2.75 \\
         3.00 & -2.75 & 11.50 \
         \end{array} \right ) . $$
  By setting 
$$  \mathbf{L} =  \left ( \begin{array}{rrr}
         1 & 0 & 0\\
        \ell_{21} &  1 &  0 \\
        \ell_{31} & \ell_{32} & 1
         \end{array} \right ),  \quad
          \mathbf{U} =  \left ( \begin{array}{rrr}
         u_{11} & u_{12} & u_{13} \\
        0  &  u_{22} & u_{23} \\
        0 & 0  & u_{33}    \end{array} \right )
        $$
 and forming the product,
find a lower triangular matrix $ \mathbf{L}$ with ones on the diagonal,
and an upper  triangular matrix $ \mathbf{U}$ such that 
$$  \mathbf{L} \mathbf{U} = \bA.$$

 This is how Gaussian elimination is implemented in MATLAB,
 the matrix $\mathbf{U} $ is the upper triangular matrix formed by GE, the
 matrix $\mathbf{L} $ contains the row mutipliers used at each stage
 of the elimination, and the matrix $\mathbf{P}$  is the permutation 
 matrix   needed  to effect the partial pivoting.
You can check your results by typing the MATLAB command sequence
\begin{verbatim}
>> A=[4,-2,3;-2,7.25,-2.75;3,-2.75,11.5], [L,U,P] = lu(A),
\end{verbatim}



\end{enumerate}
\hfill {\tiny Last modified  \today}
\vfill\eject
%%---------------------------------------------------------------------------------------

\end{document}
