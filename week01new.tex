\documentclass[11pt,a4paper]{article}

\usepackage[twoside,a4paper,hmarginratio=3:2,vmarginratio=1:1,bmargin=2.54cm]{
geometry}                % See geometry.pdf to learn the layout options. There are lots.
\geometry{a4paper}                   % ... or a4paper or a5paper or ... 
%\geometry{landscape}                % Activate for for rotated page geometry
%\usepackage[parfill]{parskip}    % Activate to begin paragraphs with an empty line rather than an inden

\usepackage{graphicx}
\usepackage{graphics}
\usepackage{amssymb}
\usepackage{amsmath}
\usepackage{epstopdf}
\usepackage{longtable}
\usepackage{lscape}
%\usepackage{savepapr}
\usepackage{fancyhdr}
\usepackage{fancybox}
\usepackage{indentfirst}
\usepackage{ifthen}
\usepackage{comment}
\usepackage{flafter}
%%%\usepackage{reenumi}
\usepackage{bibentry}
%\usepackage{hyperref}

\usepackage[latin1]{inputenc}
\usepackage{amsmath}
\usepackage{amsfonts}
%\usepackage{makeidx}
\usepackage{bm}
\usepackage{multicol}
\usepackage{color}




\newcommand{\dif}{\mathsf{d}}

%%---------------------------------------------------------------  DJS Definitions 
\def\ffrac#1#2{\leavevmode\kern.1em
\raise.5ex\hbox{\the\scriptfont0 #1}\kern-.1em
/\kern-.15em\lower.25ex\hbox{\the\scriptfont0 #2}}
\def\half{\frac{1}{2}}
\def\hhalf{\ffrac{1}{2}}
\def\bA{\mathbf{A}}
\def\bB{\mathbf{B}}
\def\bC{\mathbf{C}}
\def\bD{\mathbf{D}}
\def\bI{\mathbf{I}}
\def\bP{\mathbf{P}}
\def\bX{\mathbf{X}}
\def\tough{$\!\!\!{}^\star\>$}
\newcommand{\R}{{\mathbb{R}}}


%% Change this boolean to true to compile solutions.
\newboolean{mynotes}
\setboolean{mynotes}{true}

% Uncomment bellow to manually set notes (ie answers) to on
%\setboolean{mynotes}{true}

%%---------------------------------------------------------------------------------------
\begin{document}

\begin{center} 
{\bf Maths Problems for  CHEN10072 \\
\ifthenelse{\boolean{mynotes}}{(With solutions) \\}

 Bill Lionheart
}
\end{center}
\hrule
\smallskip



\section*{Week 1}

\begin{enumerate}

\item\label{q1167} Let
\[ \bA \,=\, \left( \begin{array}{rr} 1 &2 \\ -2 &2 \end{array} \right) \quad
\hbox{and} \quad
\bB \,=\, \left( \begin{array}{rr} -3 &1 \\ -1 &-4 \end{array} \right)
\hbox{and} \quad
\bC \,=\, \left( \begin{array}{rrr} 1 &2 & 3 \\ 0 &1 &-1\end{array} \right).\]

Compute the following matrices:
	\begin{enumerate}
	\item Calculate $\bA\bB$ and $\bB \bA$
	\item Calculate any of these matrix expressions that are possible $\bA \bC$, $\bC \bA$ $\bA+2\bB$, $\bA+\bC$. 
	\item $\bB^T \bA^T$.  Is it true that $(\bA \bB)^T= \bB^T \bA^T$?
	\item Calculate $\bC^T \bC$ and  $\bC \bC^T$ 
	\end{enumerate}

\ifthenelse{\boolean{mynotes}}{
\noindent\hrulefill

{\bf Solution}

\begin{enumerate}
	\item\[ \bA \bB = \left(
	\begin{array}{cc}
	-5 & -7 \\
	4 & -10
	\end{array}
	\right), \quad \bB\bA= \left(
	\begin{array}{cc}
	-5 & -4 \\
	7 & -10
	\end{array}
	\right),\quad \bA \bB \ne \bB \bA \]
\item 
\[\bA \bC=\left(
\begin{array}{rrr}
1  & 4  & 1\\
-2  & -2 & -8 \end{array}\right), \,
 \bA + 2 \bB=\left(
 \begin{array}{rr}
   -6 &  2 \\
   -2  &-8
 \end{array}\right) 
 \]

\item \[\bB^T \bA^T=\left(
\begin{array}{cc}
 -5 & 4 \\
 -7 & -10
\end{array}
\right)= (\bA \bB)^T\]


\item \[\bC \bC^T=\left(
\begin{array}{cc}
14  & -1 \\
-1  &  2 \end{array}
\right), \,
\bC^T\bC=\left(
\begin{array}{ccc}
   1  &  2  &  3 \\
   2  &  5  &  5\\
   3  &  5  & 10\end{array}\right)\]
\end{enumerate}

\noindent\hrulefill

}{}%



\item\label{qwrblm1}
Let 
\[ \bA =  \left( \begin{array}{r} 1 \\ 2 \\ -1 \end{array} \right) \quad
\hbox{and} \quad
\bB =  \left( \begin{array}{r} 2 \\ -1 \\ 0 \end{array} \right)
\] 
\begin{enumerate}
	\item Calculate $\bA^T \bA$.
	\item Calculate $\bA^T \bB$.
	\item What do the above mean if we think of the matrices as vectors?
	\item Calculate $\bA \bA^T$ and $\bA \bA^T \bB$.
	\item Calculate $\bA \bA^T + \bI$ where $\bI$ is the $3\times 3 $ identity matrix.

	\end{enumerate}

\ifthenelse{\boolean{mynotes}}{
\noindent\hrulefill

{\bf Solution}

\begin{enumerate}
	\item 6
	\item $0$, or $(0)$
	\item The squared length $||\bA||^2=6$ and the dot product $\bA \cdot \bB=0$ so the two vector are orthogonal. 
	\item \[  \bA \bA^T=\left(
\begin{array}{ccc}\
 1 & 2 & -1 \\
 2 & 4 & -2 \\
 -1 & -2 & 1
\end{array}
\right) \]

\item \[ \bA \bA^T + \bI= \left(
\begin{array}{ccc}
 2 & 2 & -1 \\
 2 & 5 & -2 \\
 -1 & -2 & 2
\end{array}
\right) \]
\item \[ \det( \bA \bA^T + \bI)=7 \] 

\end{enumerate}
\noindent\hrulefill	

}{}%	


\end{enumerate}
\hfill {\tiny Last modified  \today}
\vfill\eject



\end{document}
