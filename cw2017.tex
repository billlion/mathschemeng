\documentclass[12pt]{article}
%\usepackage{mathexam,amssymb,theorem}
\usepackage{amsfonts}
\usepackage {amsmath,amssymb}







%%%%%%%%%%%%%%%%%%%%%%%%%%%%%%%%%%
% Make uppercase Greek characters italic.
% Copied from latex.ltx and changed second digit from 0 (roman font)
% to 1 (math italic).
\mathchardef\Gamma="7100
\mathchardef\Delta="7101
\mathchardef\Theta="7102
\mathchardef\Lambda="7103
\mathchardef\Xi="7104
\mathchardef\Pi="7105
\mathchardef\Sigma="7106
\mathchardef\Upsilon="7107
\mathchardef\Phi="7108
\mathchardef\Psi="7109
\mathchardef\Omega="710A
%%%%%%%%%%%%%%%%%%%%%%%%%%%%%%%%%
% Define a few macros specific to this exam and this particular typist.
\def\ffrac#1#2{\leavevmode\kern.1em
\raise.5ex\hbox{\the\scriptfont0 #1}\kern-.1em
/\kern-.15em\lower.25ex\hbox{\the\scriptfont0 #2}}
\def\half{\frac{1}{2}}
\def\hhalf{\ffrac{1}{2}}
\def\bA{\mathbf{A}}
\def\bB{\mathbf{B}}
\def\bC{\mathbf{C}}
\def\bD{\mathbf{D}}
\def\bI{\mathbf{I}}
\def\bL{\mathbf{L}}
\def\bU{\mathbf{U}}
\def\bP{\mathbf{P}}
\def\bX{\mathbf{X}}
\def\tough{$\!\!\!{}^\star\>$}
\newcommand{\R}{{\mathbb{R}}}
\newcommand{\dif}{\mathsf{d}}

\begin{document}
\begin{center}{\bf\large
CHEN10072 ENGINEERING MATHEMATICS 2
\\
Coursework 2017. Due Thursday March 16th by 1400. 
}
\end{center}
\begin{center}{\bf
Answer all questions. Answers should be on paper, neatly handwritten.
}
\end{center}

\begin{enumerate}

\item 
Given
\vspace{-0.5cm}
\begin{eqnarray*}
x_1\,+\,x_2\,+\,x_3 + x_4&=&3\\
\,\,\,\,\,x_2\,+\,x_3 + 3x_4&=&2\\
x_1\,-\,2x_2\,+\,x_3 - x_4&=&4\\
\,\,\,\,\,\,x_2\,+\,x_3 + x_4&=&1
\end{eqnarray*}
            \begin{enumerate}
            \item Write the system of  equations in the form of $\bA \mathbf{x}=\mathbf{b}$.
             
            \item  Solve for $x_1,x_2,x_3,x_4$ using Gaussian elimination {\em with partial  pivoting}.
            \item At which point in the process is it clear that the matrix $\mathbf{A}$ is invertible?
            \item Check your answer to (b) using  MATLAB and attach a print-out to show how you did this.
    \end{enumerate}
	

\item
Consider the ordinary differential equation
$$ \frac{\dif y}{\dif x}\,=\,  \frac{1+3y}{y} x^2 $$
\begin{enumerate}
\item
Find the general solution in implicit form, that is an expression connecting  $x$ $y$ and an unknown constant $C$.
\item Find the particular solution satisfying the initial conditions $y(1)=1$,  in implicit form. 
\item Find the particular solution satisfying the initial conditions $y(-1)=1/3$. 
\end{enumerate}
\vspace{4cm}

\tiny{W. Lionheart and M Crabb \today}



\end{enumerate}


\end{document}
