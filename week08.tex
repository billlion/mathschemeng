\documentclass[11pt,a4paper]{article}

\usepackage[twoside,a4paper,hmarginratio=3:2,vmarginratio=1:1,bmargin=2.54cm]{
geometry}                % See geometry.pdf to learn the layout options. There are lots.
\geometry{a4paper}                   % ... or a4paper or a5paper or ... 
%\geometry{landscape}                % Activate for for rotated page geometry
%\usepackage[parfill]{parskip}    % Activate to begin paragraphs with an empty line rather than an inden

\usepackage{graphicx}
\usepackage{graphics}
\usepackage{amssymb}
\usepackage{amsmath}
\usepackage{epstopdf}
\usepackage{longtable}
\usepackage{lscape}
%\usepackage{savepapr}
\usepackage{fancyhdr}
\usepackage{fancybox}
\usepackage{indentfirst}
\usepackage{ifthen}
\usepackage{comment}
\usepackage{flafter}
%%%\usepackage{reenumi}
\usepackage{bibentry}
%\usepackage{hyperref}

\usepackage[latin1]{inputenc}
\usepackage{amsmath}
\usepackage{amsfonts}
%\usepackage{makeidx}
\usepackage{bm}
\usepackage{multicol}
\usepackage{color}

\newcommand{\dif}{\mathsf{d}}

%%---------------------------------------------------------------  DJS Definitions 
\def\ffrac#1#2{\leavevmode\kern.1em
\raise.5ex\hbox{\the\scriptfont0 #1}\kern-.1em
/\kern-.15em\lower.25ex\hbox{\the\scriptfont0 #2}}
\def\half{\frac{1}{2}}
\def\hhalf{\ffrac{1}{2}}
\def\bA{\mathbf{A}}
\def\bB{\mathbf{B}}
\def\bC{\mathbf{C}}
\def\bD{\mathbf{D}}
\def\bI{\mathbf{I}}
\def\bP{\mathbf{P}}
\def\bX{\mathbf{X}}
\def\tough{$\!\!\!{}^\star\>$}
\newcommand{\R}{{\mathbb{R}}}


%% Change this boolean to true to compile solutions.
\newboolean{mynotes}
\setboolean{mynotes}{true}

%\setboolean{mynotes}{false}

%%---------------------------------------------------------------------------------------
\begin{document}


\begin{center} 
{\bf Maths Problems for 2017 | CHEN10072 \\
Bill Lionheart, Michael Crabb\\
 March 2017}
\end{center}

\section*{Week 8}

 \begin{enumerate}
\item \label{qdjsode1}
 Consider a population of bacteria in a confined environment in which no more than 
 $B$ elements can coexist. We assume that, at the initial time $t=0$, the number of individuals
 is  much smaller than $B$ and is equal to $y_0$. We also assume that
 the growth rate of the bacteria is a positive constant  $C$.
 In this case the rate of change of the population is proportional to the number
 of existing bacteria, under the restriction that the total number cannot exceed
 $B$. This is expressed by the differential equation 
 $${\dif y\over \dif t} = Cy \left( 1-{y\over B}\right),$$
 whose solution $y=y(t)$ denotes the number of bacteria at time $t$.
 
 \begin{enumerate}
\item 
Given some initial population $y_0$ and a  step length of $h$,
write down the general update formula for computing the forward Euler 
solution  $y_{n+1}$  from the previous estimate $y_n\approx y(nh)$.
\item 
Next, write down the general update formula for computing the {\it implicit} Euler 
solution  $y^*_{n+1}$  from the previous estimate $y^*_n \approx y(nh)$.
You should find that this approach requires the solution if a 
{\it quadratic} equation at every time step.  \end{enumerate}


\ifthenelse{\boolean{mynotes}}{
\noindent\hrulefill
{\bf Solution}
\noindent\hrulefill


\begin{enumerate}
	\item
	Given a numerical estimate $y_n$ at time level $t_n=nh$ the forward Euler 
	solution $y_{n+1}\approx y(t_{n+1})$ is given by $y_{n+1}=y_n +h f(t_n, y_n)$.
	In this case $f= Cy \left( 1-{y\over B}\right)$  so we have that
	$$
	y_{n+1}=y_n  + C h \, y_n \left( 1-{y_n\over B}\right).
	$$  
\item 
	Given a numerical estimate $y^*_n$ at time level $t_n=nh$ the {\it implicit} Euler 
	solution $y^*_{n+1}\approx y(t_{n+1})$ is given by $y^*_{n+1}=y^*_n +h f(t_{n+1}, y^*_{n+1})$.
	In this case $f= Cy \left( 1-{y\over B}\right)$  so we have that
	$$
	y^*_{n+1}=y^*_n  + C h \, y^*_{n+1} \left( 1-{y^*_{n+1}\over B}\right).
	$$  
	Rearranging this gives a quadratic equation for the solution value $s=y^*_{n+1}$ :
	$$
	{Ch\over B} s^2 +(1-Ch) s - y^*_n = 0.
	$$
	\end{enumerate}
 
}{}%



\item \label{qdjsode2}

Consider  the following  linear {\it second order} differential equation
problem: find  $y(x)$ such that
$$
{\dif^2 y\over \dif x^2} +{\dif y\over \dif x} + x^2 y =0 ; \quad y(1)=0; \>\> {\dif y\over \dif x} (1)=1.
$$

\begin{enumerate}
\item
%Using the substitution $v=\dif y/\dif x$ show that this problem can be rewritten  as a system of
Using the substitution  $v=\dif y/\dif x$ show that can be rewritten as a system of
first order differential equations, and find the associated initial conditions.
\item
 Solve the differential equation numerically to find an approximation to 3 significant figures for $y(1.1)$, using (i) Forward Euler method and (ii) Backward Euler method, using $2$ steps with a step size of $h=0.05$.

\end{enumerate}

 
\ifthenelse{\boolean{mynotes}}{
\noindent\hrulefill
{\bf Solution}
\noindent\hrulefill



\begin{enumerate}
\item Using substitution we have
$$\frac{\dif v}{\dif x} + v + x^2 y=0, \frac{\dif y}{\dif x} =v$$
or
$$\frac{\dif }{\dif x} \left(\begin{array}{c} y \\ v\end{array} \right)
=\left(\begin{array}{cc} 0 & 1 \\ -x^2 & -1 \end{array} \right) \left(\begin{array}{c} y \\ v\end{array} \right)
$$
$y(1)=0, v(1)=1$
\item For general linear ODE system i.e. $\mathbf{w}'=\mathbf{F}(\mathbf{w},x)$ , $w_j=\mathbf{w}(x_j)$

FE: $\mathbf{w}_{j+1} = \mathbf{w}_{j} + h \mathbf{F}(\mathbf{w}_j,x_j)$

BE: $\mathbf{w}_{j+1} = \mathbf{w}_{j} + h \mathbf{F}(\mathbf{w}_{j+1},x_{j+1})$

Case here i.e. ODE coefficients functions of dependent variable $\mathbf{F}(\mathbf{w},x)= A(x) \mathbf{w}$

FE: $\mathbf{w}_{j+1} = \mathbf{w}_{j} + h A(x_j)\mathbf{w}_j$ so $\mathbf{w}_{j+1} = (I +hA(x_{j})) \mathbf{w}_{j}$

BE: $\mathbf{w}_{j+1} = \mathbf{w}_{j} + h A(x_{j+1})\mathbf{w}_{j+1}$ so $\mathbf{w}_{j+1} = (I -hA(x_{j+1}))^{-1} \mathbf{w}_{j}$

\begin{enumerate}
 \item  $$\left(\begin{array}{c} y \\ v\end{array} \right)_{x=1.05} = \left(   
 \left(\begin{array}{cc} 1 & 0 \\ 0  & 1\end{array} \right)
 + 0.05 \left(\begin{array}{cc} 0 & 1 \\ -1^2  & -1 \end{array} \right) 
 \right)
 \left(\begin{array}{c} y \\ v\end{array} \right)_{x=1}
 $$
 $$= \left(\begin{array}{cc} 1 & 0.05 \\ -0.05  & 0.95\end{array} \right) \left(\begin{array}{c} 0 \\ 1\end{array} \right)=\left(\begin{array}{c} 0.05 \\ 0.95\end{array} \right) $$ 
 
$$\left(\begin{array}{c} y \\ v\end{array} \right)_{x=1.1} = \left(   
 \left(\begin{array}{cc} 1 & 0 \\ 0  & 1\end{array} \right)
 + 0.05 \left(\begin{array}{cc} 0 & 1 \\ -1.05^2  & -1 \end{array} \right) 
 \right)
 \left(\begin{array}{c} y \\ v\end{array} \right)_{x=1.05}
 $$
 $$= \left(\begin{array}{cc} 1 & 0.05 \\ -0.055125  & 0.95\end{array} \right) \left(\begin{array}{c} 0.05 \\ 0.95 \end{array} \right)=\left(\begin{array}{c} 0.0975 \\ 0.900\end{array} \right) $$ 
 
 and $y(1.1)= 0.0975$ to 3 s.f.
 \item 
  $$\left(\begin{array}{c} y \\ v\end{array} \right)_{x=1.05} = \left(   
  \left(\begin{array}{cc} 1 & 0 \\ 0  & 1\end{array} \right)
  - 0.05 \left(\begin{array}{cc} 0 & 1 \\ -1.05^2  & -1 \end{array} \right) 
  \right)^{-1}
  \left(\begin{array}{c} y \\ v\end{array} \right)_{x=1}
  $$
  $$= \left(\begin{array}{cc} 1 & -0.05 \\ 0.055125  & 1.05\end{array} \right)^{-1} \left(\begin{array}{c} 0 \\ 1\end{array} \right)=\left(\begin{array}{c} 0.04749437 \\ 0.9498875\end{array} \right) $$ 
  
  $$\left(\begin{array}{c} y \\ v\end{array} \right)_{x=1.1} = \left(   
  \left(\begin{array}{cc} 1 & 0 \\ 0  & 1\end{array} \right)
  - 0.05 \left(\begin{array}{cc} 0 & 1 \\ -1.1^2  & -1 \end{array} \right) 
  \right)^{-1}
  \left(\begin{array}{c} y \\ v\end{array} \right)_{x=1.05}
  $$
  $$= \left(\begin{array}{cc} 1 & -0.05 \\ 0.0605  & 1.05\end{array} \right)^{-1} \left(\begin{array}{c} 0.04749437 \\ 0.9498875\end{array} \right)=\left(\begin{array}{c} 0.0925 \\ 0.899\end{array} \right) $$ 
  
  and $y(1.1)= 0.0925$ to 3 s.f.
\end{enumerate}
\end{enumerate}
 
}{}%



\item \label{qdjsode3}\tough
In Biochemistry, a Michaelis--Menten type process involves
 a substrate $S$, an enzyme $E$, a complex $C$ and a product $P$, and is
 summarized through the set of  reactions 
 \begin{eqnarray*}
S + E &\stackrel{c_1}{\longrightarrow}& C\\
C  &\stackrel{c_2}{\longrightarrow}& S + E\\
C &\stackrel{c_3}{\longrightarrow}& P+E.
\end{eqnarray*}
In the framework of chemical kinetics, this set of reactions may be interpreted as
a systems of ODEs as follows
  \begin{eqnarray*}
{\dif S\over \dif t}   &= &  -c_1 S E  +c_2\,C\\
{\dif E\over \dif t}   &= &  -c_1 S E  + (c_2 +c_3) \,C\\
{\dif C\over \dif t}   &= &  c_1 S E  - (c_2 +c_3) \,C\\
{\dif P\over \dif t}   &= &  c_3 \,C,
\end{eqnarray*}
where  $S(t)$,   $E(t)$,  $C(t)$ and  $P(t)$ denote the concentrations of substrate,
enzyme, complex and product, respectively,  at time $t$. 

 \begin{enumerate}
\item 
Given some initial values $S_0$,   $E_0$,  $C_0$ and  $P_0$ and a step length $h$,
write down a set of update formulas for computing the forward Euler solution
 $S_{n+1}$,   $E_{n+1}$,  $C_{n+1}$ and  $P_{n+1}$   from the previous time
 solution estimates  $S_{n}$,   $E_{n}$,  $C_{n}$ and  $P_{n}$. 
 \item
 Next, write down the general update formula for computing the {\it implicit} Euler 
solution
$S^*_{n+1}$,   $E^*_{n+1}$,  $C^*_{n+1}$ and  $P^*_{n+1}$   from the previous 
time solution  estimates.  You will  find that this approach requires the solution of a 
system of linear equations at every time step.  
\end{enumerate}


\ifthenelse{\boolean{mynotes}}{
\noindent\hrulefill
{\bf Solution}
\noindent\hrulefill

\begin{enumerate}
	\item
	Given  numerical estimates $S_{n}$,   $E_{n}$,  $C_{n}$ and  $P_{n}$
          at time level $t_n=nh$ the forward Euler 
	solution is given by 
	\begin{eqnarray*}
       S_{n+1}   &= &  S_{n} +h( -c_1 S_n E_n  +c_2\,C_n)\\
       E_{n+1}  &= &   E_{n}  +h(  -c_1 S_n E_n  + (c_2 +c_3) \,C_n)\\
      C_{n+1}  &= &   C_{n} +h( c_1 S_n E_n  - (c_2 +c_3) \,C_n)\\
      P_{n+1}   &= &   P_{n}+h  c_3 \,C_n.
\end{eqnarray*}

\item 
	Given  numerical estimates $S^*_{n}$,   $E^*_{n}$,  $C^*_{n}$ and  $P^*_{n}$
          at time level $t_n=nh$  the {\it implicit} Euler solution is given by 
	\begin{eqnarray*}
       S^*_{n+1}   &= &  S^*_{n} +h( -c_1 S^*_{n+1} E^*_{n+1}  +c_2\,C^*_{n+1})\\
       E^*_{n+1}  &= &   E^*_{n}  +h(  -c_1 S^*_{n+1} E^*_{n+1}  + (c_2 +c_3) \,C^*_{n+1})\\
      C^*_{n+1}  &= &   C^*_{n} +h( c_1 S^*_{n+1} E^*_{n+1}  - (c_2 +c_3) \,C^*_{n+1})\\
      P^*_{n+1}   &= &   P^*_{n}+h  c_3 \,C^*_{n+1}.
\end{eqnarray*}
This is a system of {\it quadratic equations}. 
Rearranging leads to a linear equation system:	
\begin{eqnarray*}
      (1+ h\, c_1 E^*_{n+1} )S^*_{n+1}   - c_2 h \,C^*_{n+1} &= &  S^*_{n} \\
     (1+ h\, c_1 S^*_{n+1} )  E^*_{n+1}  - h\, (c_2 +c_3) \,C^*_{n+1} &= &   E^*_{n}  \\
     (h \, c_1 S^*_{n+1}) E^*_{n+1} +  (1  +h\,  c_2 + h\,c_3) C^*_{n+1}  &= &   C^*_{n}   \\
      -h\,  c_3 \,C^*_{n+1} +   P^*_{n+1}  &= &   P^*_{n} \end{eqnarray*}
     except that  the coefficient  matrix depends on the unknown values
$$\left ( \begin{array}{cccc}
         1+ h  c_1 E^*_{n+1} & 0 &  - c_2 h  & 0 \\
        0 &  1+ h  c_1 S^*_{n+1}  & - h  (c_2 +c_3) & 0 \\
        0 & h  c_1 S^*_{n+1} & 1  +h  c_2 + h c_3 & 0 \\
         0 &  0 &   -h  c_3 &  1 \end{array} \right )
   \left ( \begin{array}{c}
         S^*_{n+1} \\ E^*_{n+1} \\ C^*_{n+1} \\   P^*_{n+1} 
            \end{array} \right) =
   \left ( \begin{array}{r}
        S^*_{n} \\ E^*_{n} \\ C^*_{n} \\   P^*_{n} 
            \end{array} \right) .$$
To solve this system we would need to use some type of nonlinear iteration. 
This will require solving systems of linear equations with ``frozen'' 
matrix coefficients  at every time step. 
	\end{enumerate}
 
}{}%





\end{enumerate}
\vfill\eject
\end{document}
