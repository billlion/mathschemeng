\documentclass[11pt,a4paper]{article}

\usepackage[twoside,a4paper,hmarginratio=3:2,vmarginratio=1:1,bmargin=2.54cm]{
geometry}                % See geometry.pdf to learn the layout options. There are lots.
\geometry{a4paper}                   % ... or a4paper or a5paper or ... 
%\geometry{landscape}                % Activate for for rotated page geometry
%\usepackage[parfill]{parskip}    % Activate to begin paragraphs with an empty line rather than an inden

\usepackage{graphicx}
\usepackage{graphics}
\usepackage{amssymb}
\usepackage{amsmath}
\usepackage{epstopdf}
\usepackage{longtable}
\usepackage{lscape}
%\usepackage{savepapr}
\usepackage{fancyhdr}
\usepackage{fancybox}
\usepackage{indentfirst}
\usepackage{ifthen}
\usepackage{comment}
\usepackage{flafter}
%%%\usepackage{reenumi}
\usepackage{bibentry}
%\usepackage{hyperref}

\usepackage[latin1]{inputenc}
\usepackage{amsmath}
\usepackage{amsfonts}
%\usepackage{makeidx}
\usepackage{bm}
\usepackage{multicol}
\usepackage{color}

\newcommand{\dif}{\mathsf{d}}

%%---------------------------------------------------------------  DJS Definitions 
\def\ffrac#1#2{\leavevmode\kern.1em
\raise.5ex\hbox{\the\scriptfont0 #1}\kern-.1em
/\kern-.15em\lower.25ex\hbox{\the\scriptfont0 #2}}
\def\half{\frac{1}{2}}
\def\hhalf{\ffrac{1}{2}}
\def\bA{\mathbf{A}}
\def\bB{\mathbf{B}}
\def\bC{\mathbf{C}}
\def\bD{\mathbf{D}}
\def\bI{\mathbf{I}}
\def\bP{\mathbf{P}}
\def\bX{\mathbf{X}}
\def\tough{$\!\!\!{}^\star\>$}
\newcommand{\R}{{\mathbb{R}}}


%% Change this boolean to true to compile solutions.
\newboolean{mynotes}
\setboolean{mynotes}{true}

%%---------------------------------------------------------------------------------------
\begin{document}

\begin{center} 
{\bf Maths Problems for 2011--12 | CHEN10072 \\
David Silvester \\
17 January 2012}
\end{center}

\hrule
\smallskip
\section*{week 14}



\begin{enumerate}

\item\label{qdjs1v}
Given two vectors
 $$\mathbf{u}= \left ( \begin{array}{r} 2\\ 0 \\ 1 \end{array} \right ) , \quad
  \mathbf{v}=  \left ( \begin{array}{r} 3\\ 1 \\ -3 \end{array} \right ).$$  
 \begin{enumerate}
\item Find the magnitudes of $\mathbf{u}$ and $\mathbf{v}$. 
\item Find the unit vectors having the same direction as $\mathbf{u}$ and $\mathbf{v}$.
\item Find the magnitude of $\mathbf{u}+ \mathbf{v}$ and compare it with
$|\mathbf{u}| + |\mathbf{v}|$.  Explain what you discover and give a reason why.
\end{enumerate}
\ifthenelse{\boolean{mynotes}}{
\noindent\hrulefill
{\bf Solution}
\noindent\hrulefill
}{}%
	
\item\label{qdjs2v}
Given two vectors
 $$\mathbf{u}= \left ( \begin{array}{r} 1\\ -2 \\ 3 \end{array} \right ) , \quad
      \mathbf{v}=  \left ( \begin{array}{r} 3\\ 1 \\ 1\end{array} \right ).$$  
Find the dot product $\mathbf{u}\cdot \mathbf{v}$ and the angle
between  $\mathbf{u}$ and  $\mathbf{v}$. 
\ifthenelse{\boolean{mynotes}}{
\noindent\hrulefill
{\bf Solution}
\noindent\hrulefill
}{}%




\item\label{qdjs3v}
Given
  $$\mathbf{u}= \left ( \begin{array}{c} t\\ 2-t^2 \end{array} \right ). $$
 Find the values of $t$  for which   $\mathbf{u}$ is perpendicular to the vector
  $\mathbf{v}= \left ( \begin{array}{r} 1\\ -1 \end{array} \right ). $\\
Can you illustrate your result graphically? 

\ifthenelse{\boolean{mynotes}}{
\noindent\hrulefill
{\bf Solution}
\noindent\hrulefill
}{}%


\item\label{qdjs4v}\tough
Let 
  $$\mathbf{u}= \left ( \begin{array}{c} u_1\\ u_2\\ u_3 \end{array} \right )$$
 denote  a general vector in $\R^3$.
 Show that $\mathbf{u}$ is equal to the zero vector if and only if the 
 dot product of  $\mathbf{u}$ with itself is zero. Does this 
 characterization  of the zero vector extend to  vectors $\mathbf{u} \in \R^{10}\,$? 
\ifthenelse{\boolean{mynotes}}{
\noindent\hrulefill
{\bf Solution}
\noindent\hrulefill
}{}%


\item\label{qdjs5v}\tough
Show that, for all vectors $\mathbf{u},  \mathbf{v} \in \R^3$, 
 \begin{enumerate}
\item $| \mathbf{u} +   \mathbf{v}|^2 + | \mathbf{u} -   \mathbf{v}|^2 =
2 | \mathbf{u}|^2 + 2 | \mathbf{v}|^2 .$
\item $ {1\over 4} \left( | \mathbf{u} +   \mathbf{v}|^2 -  | \mathbf{u} -   \mathbf{v}|^2 \right)=
\mathbf{u}\cdot \mathbf{v}.$
 \end{enumerate}
Do these two results extend to  vectors $\mathbf{u},  \mathbf{v} \in \R^{10}\,$? 

\ifthenelse{\boolean{mynotes}}{
\noindent\hrulefill
{\bf Solution}
\noindent\hrulefill
}{}%
\end{enumerate}

\vfill\eject
%%---------------------------------------------------------------------------------------
\section*{week 15}
\begin{enumerate}

\item\label{q1167} Let
\[ \bA \,=\, \left( \begin{array}{rr} 1 &2 \\ -2 &2 \end{array} \right) \quad
\hbox{and} \quad
\bB \,=\, \left( \begin{array}{rr} -3 &1 \\ -1 &-4 \end{array} \right).\]

Compute the following matrices:
	\begin{enumerate}
	\item $\bA+ 2\bB$.  Is it true that $ \bA + 2\bB= 2\bB + \bA$?
	\item $\bA^T+  \bB^T$. Is it true that $( \bA + \bB)^T = \bA^T + \bB^T$?
	\item $\bA \bB$ and $\bB \bA$.    Is it true that $ \bA \bB= \bB \bA$?
	\item $\bB^T \bA^T$.  Is it true that $(\bA \bB)^T= \bB^T \bA^T$?
	\end{enumerate}

\ifthenelse{\boolean{mynotes}}{
\noindent\hrulefill

{\bf Solution}

\noindent\hrulefill

}{}%

\item\label{qdjsp1}
A {\it permutation} matrix  $\bP_{ij}$ is generated by interchanging rows (or colums) 
of an {\it identity} matrix $\bI$, e.g.
$$  \bP_{12} =  \left ( \begin{array}{ccc}  0 & 1 & 0\\  1 & 0 & 0\\  0 & 0 & 1 \end{array} \right ),  \quad
  \bP_{13} =  \left ( \begin{array}{ccc}  0 & 0 & 1\\  0 & 1 & 0\\  1 & 0 & 0 \end{array} \right ).  $$

Let $\bA$ be the matrix
$$  \bA =  \left ( \begin{array}{ccc}  11 & 12 & 13\\  21 & 22 & 23 \\  31 & 32 & 33 \end{array} \right ).$$
\begin{enumerate}
\item Compute the products $\bP_{12} \bA$,  $\bP_{13} \bA$,   $\bA\bP_{12}$ and  $\bA\bP_{13}$.
What do you observe?
\item  Show that  $\bP_{12} \bP_{12}^T = \bI$.  (This means that $\bP_{12}^T$ is the inverse of   $\bP_{12}$---this
property characterizes a  permutation matrix.)
\item Show that the matrix $\bP= \bP_{12} \bP_{13} $ is also a permutation matrix. (Compute $\bP$
explicitly  and show that $\bP \bP^T=\bI$.)
\item Compute  the determinant of $\bP_{12}$ and that of $\bP_{13}$.
\item\tough Use the fact that  $\hbox{det}(\bA \bB) =  \hbox{det}(\bA) \times \hbox{det}(\bB)$ to
prove that the determinant of a permutation matrix is $\pm 1$. \emph{Hint: 
use the fact that transposing a matrix does not change its determinant.}


\end{enumerate}
\ifthenelse{\boolean{mynotes}}{
\noindent\hrulefill
{\bf Solution}
\noindent\hrulefill
}{}%


\end{enumerate}
\vfill\eject
%%---------------------------------------------------------------------------------------
\section*{week 16}
\begin{enumerate}
 

\item\label{q1171}  If
 \[\bA \,= \left( \begin{array}{rr} 1&2\\-2&4 \end{array} \right),\] calculate
$\bA^{-1}$.   By direct calculation show that
 $\bA\bA^{-1} =\bA^{-1}\bA=\bI $.

\ifthenelse{\boolean{mynotes}}{
\noindent\hrulefill
{\bf Solution}
\noindent\hrulefill
}{}%


\item\label{q1172} Consider the linear equations
\begin{eqnarray*}
3x\,+\,2y&=&1\\ 
-2x\,+\,5y&=&-7.
\end{eqnarray*}
	\begin{enumerate}
	\item Write this equations in matrix form, i.e. $\bA \mathbf{x}=\mathbf{b}$.
	\item Calculate $\bA^{-1}$ and, by direct calculation, show $\bA \bA^{-1} = \bA^{-1} \bA =\bI $.
	\item Use $\bA^{-1}$ to obtain the values of $x$ and $y$.
	\item Check your working by showing your answer fits the original equations.
	\end{enumerate}

\ifthenelse{\boolean{mynotes}}{
\noindent\hrulefill
{\bf Solution}
\noindent\hrulefill
}{}%

\item\label{qdjsp2}
\begin{enumerate}
\item
Evaluate the following  $2\times 2$  determinants:
 $$ d_1= \left| \begin{array}{rr} 1&2 \\ -2 & 5 \end{array} \right| , \qquad
	d_2=  \left| \begin{array}{rr}  \ffrac{5}{9} &- \ffrac{2}{9}  \\  \ffrac{2}{9}  &  \ffrac{1}{9} \end{array} \right|. $$ 
How are  $d_1$ and $d_2$ related and why?
\item\tough
Next, evaluate the following  $3\times 3$ determinants:
 $$  d_3 = \left| \begin{array}{rrr} 1&-2&1 \\ 2 & 3 & -1\\-1&-2&2 \end{array} \right|, \qquad
	d_4=  \left| \begin{array}{rrr}2 & 3 & -1\\  1&-2&1 \\ -1&-2&2 \end{array} \right|. $$ 
How are  $d_3$ and $d_4$  related and why? \emph{Hint:  Use the properties of permutation matrices.}
\end{enumerate}
\ifthenelse{\boolean{mynotes}}{
\noindent\hrulefill
{\bf Solution}
\noindent\hrulefill
}{}%



\item\label{q1205}\tough
Suppose that you have square matrices, $\bA, \bB$ and $\bC$, with the properties
$\bA\bB=\bI$ and $\bC\bA =\bI$, where $\bI$ is the identity matrix.
Use this information to prove that $\bC = \bB$.

You may assume that  $\bD \bI = \bI\bD = \bD$  for any square matrix $\bD$. You may
not assume that $\bA \bA^{-1} = \bA^{-1} \bA = \bI$. 
The  point of the question is to establish that the
left-hand and right-hand inverses are identical and the above  statement assumes this.
\emph{Hint: Do not use  inverse matrices in this question. 
Also do not look at matrix components --- consider
matrices as single objects and make use of their multiplication properties. 
The proof is just a few lines long!}


\ifthenelse{\boolean{mynotes}}{
\noindent\hrulefill

{\bf Solution}

\noindent\hrulefill

}{}%

\end{enumerate}

\vfill\eject
%%---------------------------------------------------------------------------------------
\section*{week 17}

\begin{enumerate}

\item\label{q1206}
Potassium chromate is recovered from a methanol/water solution by evaporation, crystallisation
and filtration. The total mass flows, in units of kg~h$^{-1}$, is $E$ for evaporation, $F$ for filtration and $C$
for crystallisation. The material balance equations for potassium chromate, water and methanol are
\begin{eqnarray*}
1125&=&1.06C\,+\,0.3F\\
2250&=&0.75E\,+\,0.08C\,+\,0.4F\\
1125&=&0.25E\,+\,0.06C\,+\,0.3F .
\end{eqnarray*}
\begin{enumerate}
 \item Write the system of  equations in the form 
       $ \bA \mathbf{x}=\mathbf{b}, $
        where {\bf A} is a $3\times 3$ matrix, {\bf x} is a column vector of the unknown quantities and {\bf b}
is a column vector with known coefficients.
        \item Apply  Gaussian elimination to show that $E$, $C$ and $F$ satisfy the upper triangular system
       $$\left ( \begin{array}{rrrr}
         1.06 & 0 & 0.3  \\
        0 &  0.75& 0.37736 \\
        0  & 0 & 0.15723
         \end{array} \right )
   \left ( \begin{array}{c}
        C \\ E \\ F
            \end{array} \right) =
   \left ( \begin{array}{r}
        1125.00 \\ 2165.10 \\ 339.62  
            \end{array} \right) .$$
          \item  Hence calculate $E$, $C$ and $F$ using  back-substitution.
           \item  Calculate the determinant  of the matrix $\bA$. 
           \emph{Hint: Adding multiples of one row to another does not change the determinant of a matrix.}

       \end{enumerate}

\item\label{qdjsx1} 
It is known  that the number of floating point operations (flop) required to solve
an $n\times n$ system of linear equations using Gaussian elimination
is approximately $2 n^3 /3$ for large values of $n$. Estimate the 
size of the biggest system that 
can be solved for \pounds 10,000 if it costs \pounds 1,000 an hour to
use the IBM Roadrunner supercomputer which has a calculation rate of $10^{15}$ flop/sec. 
(This is called a {\it petaflop}. Conventional laptop computers run  at  a  {\it gigaflop},  that is $10^{9}$ flop/sec.)

\item\label{q1207} Consider the linear equation system:
\begin{eqnarray*}
x\,+\,y\,+\,z&=&3\\
x\,-\,y\,+\,2z&=&2\\
2y\,-\,z&=&5
\end{eqnarray*}
            \begin{enumerate}
            \item Write the system of  equations in the form of $\bA \mathbf{x}=\mathbf{b}$.
            \item  Solve for $x,y,z$ using Gaussian elimination.
             Can you explain this result? 
            \end{enumerate}

\ifthenelse{\boolean{mynotes}}{
\noindent\hrulefill
{\bf Solution}
\noindent\hrulefill
}{}%


\item\label{qdjsx2} \tough
Consider the coefficient matrix
 $$ \bA = \left ( \begin{array}{rrrr}
         2 & 4 & -2  \\
        2 &  8& 4 \\
        1  & -2 & -1
         \end{array} \right )$$
   Using Gaussian elimination,  solve the three distinct systems 
  $$ \bA  \mathbf{x}_1=\mathbf{e}_1,\qquad  \bA  \mathbf{x}_2=\mathbf{e}_2, 
   \qquad\bA  \mathbf{x}_3=\mathbf{e}_3,\qquad\qquad\qquad\qquad\qquad $$
  where $\mathbf{e}_1=  \left[ \begin{array}{c} 1 \\ 0\\ 0 \end{array} \right]$, 
   $\mathbf{e}_2=  \left[ \begin{array}{c} 0 \\ 1\\ 0 \end{array} \right]$
    $\mathbf{e}_3=  \left[ \begin{array}{c} 0 \\ 0\\ 1 \end{array} \right]$, \\[2ex]
  so as to  compute the columns of the matrix   $\bX$ satisfying $\bA \bX = \bI$.  (Note that this is
  clever way to compute the inverse matrix $\bX= \bA^{-1}$.)  


\ifthenelse{\boolean{mynotes}}{
\noindent\hrulefill
{\bf Solution}
\noindent\hrulefill
}{}%



\end{enumerate}

\vfill\eject
%%---------------------------------------------------------------------------------------

\section*{week 18}
\begin{enumerate}



\item\label{qdjsx3}
A paint company is trying to use up excess quantities of four shades
of green paint by mixing them to form a more popular shade. One gallon
of the new paint will be made up of $x_1$ gallons of paint~1, $x_2$
gallons of paint~2 etc. Each of the paints is made up of four
pigments. If each number repesents a percentage, 
the mixture giving the more popular shade is the solution of 
the system of equations
$$\left ( \begin{array}{rrrr}
         0 & 80 & 10 & 10 \\
        80 &  0 & 30 & 10 \\
        16 & 20 & 60 & 72 \\
         4 &  0 &  0 &  8 \end{array} \right )
   \left ( \begin{array}{c}
        x_1 \\ x_2 \\ x_3 \\ x_4 
            \end{array} \right) =
   \left ( \begin{array}{r}
        27 \\ 40 \\ 31 \\ 2 
            \end{array} \right) .$$
Find the optimal mixture by solving the system using Gaussian elimination 
with {\it partial pivoting}. (This means that you interchange rows at every step
to put the biggest number in modulus  on the diagonal. In this example,
 at the first stage of the elimination you would interchange row one with row two.)


You can check your results   in MATLAB (see also the following  exercise) by typing the command sequence
\begin{verbatim}
>> A=[0,80,10,10;80,0,30,10;16,20,60,72;4,0,0,8],  [L,U,P] = lu(A),
>> b=[27;40;31;2], x=A\b, 
\end{verbatim}


\item\label{qdjsx4}
Given a matrix
$$\bA= \left ( \begin{array}{rrr}
         4.00 & -2.00 & 3.00\\
        -2.00 &  7.25 & -2.75 \\
         3.00 & -2.75 & 11.50 \
         \end{array} \right ) . $$
  By setting 
$$  \mathbf{L} =  \left ( \begin{array}{rrr}
         1 & 0 & 0\\
        \ell_{21} &  1 &  0 \\
        \ell_{31} & \ell_{32} & 1
         \end{array} \right ),  \quad
          \mathbf{U} =  \left ( \begin{array}{rrr}
         u_{11} & u_{12} & u_{13} \\
        0  &  u_{22} & u_{23} \\
        0 & 0  & u_{33}    \end{array} \right )
        $$
 and forming the product,
find a lower triangular matrix $ \mathbf{L}$ with ones on the diagonal,
and an upper  triangular matrix $ \mathbf{U}$ such that 
$$  \mathbf{L} \mathbf{U} = \bA.$$

 This is how Gaussian elimination is implemented in MATLAB,
 the matrix $\mathbf{U} $ is the upper triangular matrix formed by GE, the
 matrix $\mathbf{L} $ contains the row mutipliers used at each stage
 of the elimination, and the matrix $\mathbf{P}$  is the permutation 
 matrix   needed  to effect the partial pivoting.
You can check your results by typing the MATLAB command sequence
\begin{verbatim}
>> A=[4,-2,3;-2,7.25,-2.75;3,-2.75,11.5], [L,U,P] = lu(A),
\end{verbatim}



\end{enumerate}

\vfill\eject
%%---------------------------------------------------------------------------------------
\section*{week 19}

\begin{enumerate}


\item\label{q1188} Solve analytically the following differential equations:
	\begin{enumerate}
	\item $\displaystyle{ \frac{\dif y}{\dif x}\,=\, -x^2y}$ ; \ \  $y=1$ when $x=0$.
	\item $\displaystyle{ \frac{\dif y}{\dif x}\,=\, -y^2\sin x}$ ; \ \  $y=1$ when $x=\pi/2$.
	\item $\displaystyle{ \frac{\dif y}{\dif x}\,=\, -x(4\,+\,y^2)}$ ; \ \ $y=2$ when $x=0$.
	\end{enumerate}


\ifthenelse{\boolean{mynotes}}{
\noindent\hrulefill

{\bf Solution}

\noindent\hrulefill

}{}%

\item\label{1189} Consider the second order reaction between species A and B,

\begin{center}
$\bA \> + \> \bB \quad \longrightarrow\quad$ {\bf Products}

\end{center}

with initial concentrations [A]$_0$ and [B]$_0$ respectively.

	\begin{enumerate}
	\item Give an argument to show that
	[B]$\,=\,$ [B]$_0$ + [A]$\,-\,$[A]$_0$.
	\item Solve the differential equation \[ \frac{\dif \mathrm{[A]}}{\dif t}\,=\, -k_2 \mathrm{[A][B]}\]
	to obtain the result
	\[\mathrm{[A]}\,=\,\frac{c\mathrm{[A]_0}\exp{(-ck_2\,t)}}{c\,+\,
\mathrm{[A]_0}\,-\,\mathrm{[A]_0}\exp(-ck_2\,t)}\] where $c= \mathrm{[B]}_0\,-\,\mathrm{[A]_0}$.
	\end{enumerate}


\ifthenelse{\boolean{mynotes}}{
\noindent\hrulefill

{\bf Solution}

\noindent\hrulefill

}{}%
	


\item\label{q1193} 
Solve the following linear ODEs by finding an  integrating factor:
	\begin{enumerate}
	\item  $\displaystyle{ \frac{\dif y}{\dif x}\,+\,y\,=\,\exp{(-x)}}$ ; \ \  $y=2\> \mathrm{when} \> x=0$.
	\item  $\displaystyle{ \frac{\dif y}{\dif x}\,+\,y \cos x\,=\,\cos{x}}$ ; \ \  $y=1\> \mathrm{when} \> x=0$.
	\item\tough  $\displaystyle{ \frac{\dif y}{\dif x}\,+\,\frac{y}{x}\,=\,\sin{x}}$ ; \ \ $y=1\> \mathrm{when} \>x=0$.
	\end{enumerate}

\ifthenelse{\boolean{mynotes}}{
\noindent\hrulefill

{\bf Solution}

\noindent\hrulefill

}{}%

\end{enumerate}
\vfill\eject
%%---------------------------------------------------------------------------------------

\section*{week 20}

%Numerical solution of a scalar nonlinear ODE. Exercises will show that 
%explicit approximation is easy and that implicit approximation leads to 
%a nonlinear equation to be solved at every time step.
 
\begin{enumerate}
 \item \label{qdjsode1}
 We would like to  to solve the differential equation
$$
{\dif y\over \dif t} = -10(t-1)y
$$
with the given  initial condition:
$y(0)=e^{-5}$. 
{A reference  solution to this problem can  be
plotted in MATLAB  by  typing the command sequence}  \\[1ex]
{\tt tt=0:0.01:2; ex=exp(-5*(tt-1).*(tt-1));plot(tt,ex,'-k')}

\begin{enumerate}
\item 
Find the analytic solution $y(t)$, and evaluate it at\\
$t=0.2,0.4, 0.6, 0.8, 1.0, 1.2$ and  $1.4$.


\item
Compute the forward Euler solution by  taking {\it seven} steps of the method
with a step length of $h=0.2$. (You will need a calculator to do this.)
Generate a table which
compare the analytic solution with the numerical solution. At which
time $t$ do you find the biggest error ?

\item
Compute a more accurate  forward Euler solution  by  taking  four steps with
a  much smaller step length of $h=0.05$. Compare this solution
with the exact solution  $y(0.2)$ and compare the accuracy of this  solution
with the first time step result  obtained in (b).
\end{enumerate}

Note that  this differential equation  is not as benign as
it looks; if one wants to solve it over a long time interval,
sophisticated numerical methods (like those that are built into MATLAB)
 are needed. A simple example of a sophisticated method is
 the implicit Euler method discussed in the lectures. 
\end{enumerate}

  
 
 %\begin{enumerate}
%\item\label{q1187} 
%Carry out one step of the second order Runge-Kutta method (i.e. calculate $y(h)$) for
%
%\[ \frac{\dif y}{\dif x}\,=\, -xy\,+\,1\]
%with the boundary condition $y=1$ when $x=0$.  
%Take the step $h=0.1$ and quote your answers to four decimal places.
%\end{enumerate}

\vfill\eject
%%---------------------------------------------------------------------------------------


\section*{week 21}


\begin{enumerate}


 \item \label{qdjsode4}
Compute the {\it implicit} (or {backward}) Euler solution  to the model problem 
$$
{\dif y\over \dif t} = -10(t-1)y
$$
with the given  initial condition:
$y(0)=e^{-5}$, 
by  taking  four steps with
a  step length of $h=0.05$. Compare this solution
with the exact solution  $y(0.2)$ and the forward Euler 
solution computed previously.



\item \label{qdjsode2}
 Consider a population of bacteria in a confined environment in which no more than 
 $B$ elements can coexist. We assume that, at the initial time $t=0$, the number of individuals
 is  much smaller than $B$ and is equal to $y_0$. We also assume that
 the growth rate of the bacteria is a positive constant  $C$.
 In this case the rate of change of the population is proportional to the number
 of existing bacteria, under the restriction that the total number cannot exceed
 $B$. This is expressed by the differential equation 
 $${\dif y\over \dif t} = Cy \left( 1-{y\over B}\right),$$
 whose solution $y=y(t)$ denotes the number of bacteria at time $t$.
 
 \begin{enumerate}
\item 
Given some initial population $y_0$ and a  step length of $h$,
write down the general update formula for computing the forward Euler 
solution  $y_{n+1}$  from the previous estimate $y_n\approx y(nh)$.
\item 
Next, write down the general update formula for computing the {\it implicit} Euler 
solution  $y^*_{n+1}$  from the previous estimate $y^*_n \approx y(nh)$.



 \end{enumerate}
 
Note that  if a general  nonlinear ODE is to be solved numerically using an $implicit$ time stepping
method then a nonlinear equation must be solved at every timestep!


\item \label{qdjsode3}\tough
In Biochemistry, a Michaelis--Menten type process involves
 a substrate $S$, an enzyme $E$, a complex $C$ and a product $P$, and is
 summarized through the set of  reactions 
 \begin{eqnarray*}
S + E &\stackrel{c_1}{\longrightarrow}& C\\
C  &\stackrel{c_2}{\longrightarrow}& S + E\\
C &\stackrel{c_3}{\longrightarrow}& P+E.
\end{eqnarray*}
In the framework of chemical kinetics, this set of reactions may be interpreted as
a systems of ODEs as follows
  \begin{eqnarray*}
{\dif S\over \dif t}   &= &  -c_1 S E  +c_2\,C\\
{\dif E\over \dif t}   &= &  -c_1 S E  + (c_2 +c_3) \,C\\
{\dif C\over \dif t}   &= &  c_1 S E  - (c_2 +c_3) \,C\\
{\dif P\over \dif t}   &= &  c_3 \,C,
\end{eqnarray*}
where  $S(t)$,   $E(t)$,  $C(t)$ and  $P(t)$ denote the concentrations of substrate,
enzyme, complex and product, respectively,  at time $t$. 

 \begin{enumerate}
\item 
Given some initial values $S_0$,   $E_0$,  $C_0$ and  $P_0$ and a step length $h$,
write down a set of update formulas for computing the forward Euler solution
 $S_{n+1}$,   $E_{n+1}$,  $C_{n+1}$ and  $P_{n+1}$   from the previous time
 solution estimates  $S_{n}$,   $E_{n}$,  $C_{n}$ and  $P_{n}$. 
 \item
 Next, write down the general update formula for computing the {\it implicit} Euler 
solution
$S^*_{n+1}$,   $E^*_{n+1}$,  $C^*_{n+1}$ and  $P^*_{n+1}$   from the previous 
time solution  estimates. You should find that this requires the solution of a 
{\it system of nonlinear equations} at every time step. 

\end{enumerate}

\end{enumerate}



\vfill\eject
%%---------------------------------------------------------------------------------------
\section*{week 22}

 
 %\noindent
%Numerical solution of a system of linear ODEs. Exercises will show that 
%explicit approximation is easy and that implicit approximation leads to 
%a system of linear equations to be solved at every time step.
\begin{enumerate}
\item\label{qdjsn1} 
Verify that the nonlinear equation 
$${x}^{1/2} \sin (x) = 1$$ has at
least one solution in the interval (1.1,1.3). Hence find a solution  
which is accurate to two decimal places using
\begin{enumerate}
%\renewcommand{\theenumii}{\roman{enumii}}
\item an interval halving method,
\item a Newton iteration method (with initial guess $x_0 = 1.2$),
\end{enumerate}

\ifthenelse{\boolean{mynotes}}{
\noindent\hrulefill
{\bf Solution}
\noindent\hrulefill
}{}%

\item\label{qdjsn2}\tough 
Derive a Newton method to calculate the fifth root of a given real 
number, $s$ say. Use your method to calculate $5^{1/5}$ correct to 
four decimal places.
\ifthenelse{\boolean{mynotes}}{
\noindent\hrulefill
{\bf Solution}
\noindent\hrulefill
}{}%

\end{enumerate}

%\begin{enumerate}
%\item\label{qdjs5} 
%As a group, generate a flow diagram for a MATLAB function to
%find the roots of three coupled nonlinear equations using
%Newton's method.
%\end{enumerate}


\end{document}
