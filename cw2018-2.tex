\documentclass[12pt]{article}
%\usepackage{mathexam,amssymb,theorem}
\usepackage{amsfonts}
\usepackage {amsmath,amssymb}







%%%%%%%%%%%%%%%%%%%%%%%%%%%%%%%%%%
% Make uppercase Greek characters italic.
% Copied from latex.ltx and changed second digit from 0 (roman font)
% to 1 (math italic).
\mathchardef\Gamma="7100
\mathchardef\Delta="7101
\mathchardef\Theta="7102
\mathchardef\Lambda="7103
\mathchardef\Xi="7104
\mathchardef\Pi="7105
\mathchardef\Sigma="7106
\mathchardef\Upsilon="7107
\mathchardef\Phi="7108
\mathchardef\Psi="7109
\mathchardef\Omega="710A
%%%%%%%%%%%%%%%%%%%%%%%%%%%%%%%%%
% Define a few macros specific to this exam and this particular typist.
\def\ffrac#1#2{\leavevmode\kern.1em
\raise.5ex\hbox{\the\scriptfont0 #1}\kern-.1em
/\kern-.15em\lower.25ex\hbox{\the\scriptfont0 #2}}
\def\half{\frac{1}{2}}
\def\hhalf{\ffrac{1}{2}}
\def\bA{\mathbf{A}}
\def\bB{\mathbf{B}}
\def\bC{\mathbf{C}}
\def\bD{\mathbf{D}}
\def\bI{\mathbf{I}}
\def\bL{\mathbf{L}}
\def\bU{\mathbf{U}}
\def\bP{\mathbf{P}}
\def\bX{\mathbf{X}}
\def\tough{$\!\!\!{}^\star\>$}
\newcommand{\R}{{\mathbb{R}}}
\newcommand{\dif}{\mathsf{d}}


\pagenumbering{gobble} % no page number
\begin{document}
\begin{center}{\bf\large
CHEN10072 ENGINEERING MATHEMATICS 2
\\
Coursework 2018. Due Thursday March 15th by 1400 ESO (C62, The Mill). 
}
\end{center}
\begin{center}{\bf
Answer all questions. Answers should be on paper, neatly handwritten.
}
\end{center}

\begin{enumerate}

\item 
Given
\vspace{-0.5cm}
\begin{eqnarray*}
x_1\,+\,2x_2\,+\,x_3 + x_4&=&3\\
\,\,\,\,\,3x_2\,+\,x_3 + 3x_4&=&4\\
x_1\,-\,2x_2\,+\,3x_3 + x_4&=&-2\\
\,\,\,\,\,\,2x_2\,+\,2x_3 + 2x_4&=&3
\end{eqnarray*}
            \begin{enumerate}
            \item Write the system of  equations in the form of $\bA \mathbf{x}=\mathbf{b}$.             
            \item  Solve for $x_1,x_2,x_3,x_4$ using Gaussian elimination.
            \item At which point in the process is it clear that the matrix $\mathbf{A}$ is invertible?
            \item Check your answer to (b) using  MATLAB and attach a print-out to show how you did this.
    \end{enumerate}
	

\item
Consider the simultaneous equations
\[
y \sin x+x+2 y=1,\, y- x^2/5=1
\]
\begin{enumerate}
	\item Rearrange both equations to write $y$ in terms of $x$ and sketch or plot curves to show there is exactly one solution with  $-2 <x<2$.
	\item Show that the Jacobian matrix (matrix of partial derivatives) for the  
system is
\[
\left(
\begin{array}{cc}
y \cos x+1 & \sin x+2 \\
-2 x/5 & 1 \\
\end{array}
\right)
\]
\item Using the starting value $x=0, y=0$ use Newton's method to find a solution to the system to two decimal places.
\end{enumerate}
\vspace{3cm}
\tiny{W. Lionheart and M Crabb \today}



\end{enumerate}


\end{document}