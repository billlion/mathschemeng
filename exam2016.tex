\documentclass[12pt]{article}
%\usepackage{mathexam,amssymb,theorem}
\usepackage{CEASexam,amsfonts}
\usepackage {amsmath,amssymb}


\coursename{Engineering Mathematics 2}
\examinationdate{End of Semester  Examination}
\examinationtime{ ??? 2016}

\coursecode{CHEN10072}
\duration{One and a Half  Hours}
\rubric{Answer {\bf ALL} questions}

\calculators{\vbox{Electronic calculators may be used, provided that
they CANNOT store text. \\
 All numerical working must be shown. }}



%%%%%%%%%%%%%%%%%%%%%%%%%%%%%%%%%%
% Make uppercase Greek characters italic.
% Copied from latex.ltx and changed second digit from 0 (roman font)
% to 1 (math italic).
\mathchardef\Gamma="7100
\mathchardef\Delta="7101
\mathchardef\Theta="7102
\mathchardef\Lambda="7103
\mathchardef\Xi="7104
\mathchardef\Pi="7105
\mathchardef\Sigma="7106
\mathchardef\Upsilon="7107
\mathchardef\Phi="7108
\mathchardef\Psi="7109
\mathchardef\Omega="710A
%%%%%%%%%%%%%%%%%%%%%%%%%%%%%%%%%
% Define a few macros specific to this exam and this particular typist.
\def\ffrac#1#2{\leavevmode\kern.1em
\raise.5ex\hbox{\the\scriptfont0 #1}\kern-.1em
/\kern-.15em\lower.25ex\hbox{\the\scriptfont0 #2}}
\def\half{\frac{1}{2}}
\def\hhalf{\ffrac{1}{2}}
\def\bA{\mathbf{A}}
\def\bB{\mathbf{B}}
\def\bC{\mathbf{C}}
\def\bD{\mathbf{D}}
\def\bI{\mathbf{I}}
\def\bL{\mathbf{L}}
\def\bU{\mathbf{U}}
\def\bP{\mathbf{P}}
\def\bX{\mathbf{X}}
\def\tough{$\!\!\!{}^\star\>$}
\newcommand{\R}{{\mathbb{R}}}
\newcommand{\dif}{\mathsf{d}}

\begin{document}
\maketitle



%\section
%\sectionheader{Answer \jazzy{all} questions}


\newpage
\question
 Let
\[ \bA \,=\, \left( \begin{array}{rr} 1 & -3 \\ 2 &  0 \end{array} \right) \quad
\hbox{and} \quad
\bB \,=\, \left( \begin{array}{rr} 2 &-1 \\ 0 & 1 \end{array} \right).\]

Compute the following matrices:
	\begin{enumerate}
	\item $2\bA+ \bA\bB$.  
	\item $\bB^T \bA^T$. Is it true that $(\bB \bA)^T = \bB^T \bA^T$?
	\item $\bA (\bA +\bB)$ and $(\bB +\bA) \bA$.    Is it true that $ \bA (\bA +\bB) = (\bB +\bA) \bA$?
	\item $ \bA^{-1}$.  Is it true that $(\bA ^{-1})^T=  (\bA^T)^{-1}$?
	\marks{8}
	\end{enumerate}



\question
Evaluate the following  $3\times 3$ determinants:
 $$  d_1 = \left| \begin{array}{rrr} 1&-2&1 \\
  2 & 3 & -1\\-1&-2&2 \end{array} \right|, \qquad
	d_2=  \left| \begin{array}{rrr}
	1 & 2 & -1\\  
	-2& 3 &-2 \\
	 1&-1 & 2 \end{array} \right|. $$ 
How are  $d_1$ and $d_2$  related and why? 

\marks{4}

\question 
Given
\vspace{-0.5cm}
\begin{eqnarray*}
y\,+\,z&=&2\\
x+2z&=&4\\
x + y+z&=&-2 .
\end{eqnarray*}
            \begin{enumerate}
            \item Write the system of  equations in the form of $\bA \mathbf{x}=\mathbf{b}$.
            \item\label{q1207b}  Determine whether or not {\bf A} is singular.  
            \item  Solve for $x,y,z$ using Gaussian elimination without pivoting.\marks{8}
    \end{enumerate}
	

\question
Solve analytically the following differential equation problem: find $y(x)$ such that
$$ \frac{\dif y}{\dif x}\,=\,  y (2-y)   ; \qquad y=-1\>\hbox{ when }\> x=0. $$
\marks{4}
\question
Solve the following problem by finding an  integrating factor:  find all $y(x)$, for $0\le x < \pi/2$, such that
$$  \frac{\dif y}{\dif x}\,+\, 2y\tan x = \sin{x}   ;  $$
\noindent

\marks{5}
\newpage
\question
Compute a numerical solution to the differential equation problem 
$$ \frac{\dif y}{\dif x} =  -2 y + x    ; \qquad y=1\>\hbox{ when }\> x=0, $$
by taking two steps of the forward Euler method with 
step size $h=0.1$.
\marks{5}

\question
Write down  Newton's iteration   for finding the roots of
a function $f(x)$. Hence or otherwise, 
determine an iterative method to calculate the square root of a given real 
number, $\beta$ say. Starting with an initial guess of $x_0 =3$, run the
iteration and   calculate $\sqrt{8}$ correct to  four decimal places.
\marks{6}

%%%%%%%%%%%%%%%%%%%%%%%%%%%%% SECTION B 
%\section
%\sectionheader{Answer \jazzy{one} of the two questions}


\question

Let
$$ \bA = \left( \begin{array}{rrr} 0 & 2& 1 \\ -1 &  3 & 0 \\ -1 & 4 & 1  \end{array} \right)$$  

%   3  	2  -3
%   1   1  -1
%  -1  -2   2


\begin{enumerate}
\item 
By forming a larger matrix  from $\bA$ and the identity matrix and applying row operations, find the inverse $\bA^{-1}$.  Verify that $\bA^{-1}\bA= \bI$ using your calculated inverse.
 \null\vspace{-0.2cm} \marks{8}
\item
An alternative mthod of calculation for $\bA^{-1}$ involves calculating all the determinants of $2 \times 2$ submatrices of $\bA$. How many additions and multplications are needed for this method and that used in (a)? (Count subtractions as additions and divisions as multiplications.)
 \null\vspace{-0.2cm} \marks{2}

\end{enumerate}

\medskip
\question 
We want to solve the following  linear  {\it  second order} differential equation
problem: find  $y(t)$ such that
$$
{\dif^2 y\over \dif t^2} + 4\,y =0 ; \quad y(0)=0; \>\> {\dif y\over \dif t} (0)=1.
$$

\begin{enumerate}
\item
Show that this problem can be rewritten  as a system of
first order differential equations, and find the associated initial conditions.
\null\vspace{-0.2cm} \marks{3}

\item
 Take a  step size  of $h=0.1$  and 
compute the first two steps of the  {\it implicit} (Backward) Euler  method
 so as to approximate $y(0.2)$.  Give a reason why the {\it implicit} Euler 
 method might be preferable
 to the  {\it forward} Euler method when solving this problem.
 \null\vspace{-0.2cm} \marks{7}
 
\end{enumerate}



\finished

\end{document}
