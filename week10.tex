\documentclass[11pt,a4paper]{article}

\usepackage[twoside,a4paper,hmarginratio=3:2,vmarginratio=1:1,bmargin=2.54cm]{
geometry}                % See geometry.pdf to learn the layout options. There are lots.
\geometry{a4paper}                   % ... or a4paper or a5paper or ... 
%\geometry{landscape}                % Activate for for rotated page geometry
%\usepackage[parfill]{parskip}    % Activate to begin paragraphs with an empty line rather than an inden

\usepackage{graphicx}
\usepackage{graphics}
\usepackage{amssymb}
\usepackage{amsmath}
\usepackage{epstopdf}
\usepackage{longtable}
\usepackage{lscape}
%\usepackage{savepapr}
\usepackage{fancyhdr}
\usepackage{fancybox}
\usepackage{indentfirst}
\usepackage{ifthen}
\usepackage{comment}
\usepackage{flafter}
%%%\usepackage{reenumi}
\usepackage{bibentry}
%\usepackage{hyperref}

\usepackage[latin1]{inputenc}
\usepackage{amsmath}
\usepackage{amsfonts}
%\usepackage{makeidx}
\usepackage{bm}
\usepackage{multicol}
\usepackage{color}

\newcommand{\dif}{\mathsf{d}}

%%---------------------------------------------------------------  DJS Definitions 
\def\ffrac#1#2{\leavevmode\kern.1em
\raise.5ex\hbox{\the\scriptfont0 #1}\kern-.1em
/\kern-.15em\lower.25ex\hbox{\the\scriptfont0 #2}}
\def\half{\frac{1}{2}}
\def\hhalf{\ffrac{1}{2}}
\def\bA{\mathbf{A}}
\def\bB{\mathbf{B}}
\def\bC{\mathbf{C}}
\def\bD{\mathbf{D}}
\def\bI{\mathbf{I}}
\def\bP{\mathbf{P}}
\def\bX{\mathbf{X}}
\def\tough{$\!\!\!{}^\star\>$}
\newcommand{\R}{{\mathbb{R}}}
\newcommand{\diff}{\mathrm{d}}

%% Change this boolean to true to compile solutions.

\newboolean{mynotes}

\input{notesflag}
%\setboolean{mynotes}{true}

\begin{document}

\begin{center} 
{\bf Maths Problems for  CHEN10072 \\
\ifthenelse{\boolean{mynotes}}{(With solutions) \\}

 Bill Lionheart
}
\end{center}
\hrule
\smallskip



\section*{Week 10}
\hrule
\smallskip

This EBL really is much more ``enquiry based''. We have n't done 2nd order ODEs yet in the lecture so you get to try some things yourself.

{\em A constant coefficient 2nd order linear ODE is of the form
$$  a\frac{\diff^2 y}{\diff t^2}  + b \frac{\diff y}{\diff t} +c y = f $$
where $y(t)$ and $f(t)$ are functions of $t$ and $a\ne 0$. We will concentrate on $f=0$. There are lots of example in HELM 19.3}

\begin{enumerate}
\item Consider the ODE
$$  \frac{\diff^2y}{\diff t^2}  + \omega^2 y = 0 $$
where $\omega$ is a non zero constant.
\begin{enumerate}
\item Check that $y(t) = \sin \omega t$ and  $y(t) = \cos \omega t$ both satisfy the ODE.
\item Check that $y(t) = A \sin \omega t + B \cos \omega t$ is a solution for any $A$ and $B$.
\end{enumerate}

\item Consider the ODE
$$  \frac{\diff^2 y}{\diff t^2}  - k^2 y = 0 $$
where $k$ is a non zero constant.
\begin{enumerate}
\item Check that $y(t) = \mathrm{e}^{kt}$ and  $y(t) = \mathrm{e}^{-kt}$ both satisfy the ODE.
\item Check that $y(t) =A\mathrm{e}^{kt} + B\mathrm{e}^{-kt}$ is a solution for any $A$ and $B$.
\end{enumerate}


\item Consider the ODE
$$  \frac{\diff^2 y}{\diff t^2}  - \frac{\diff y}{\diff t} -6 y = 0 $$ 
\begin{enumerate}
\item Try a solution of the form $y=\mathrm{e}^{kt}$. Show that for it to be a solution we must have $k^2 -k -6=0$
\item Solve the quadratic and write down two solutions of the ODE.
\end{enumerate}

{\em The equation for $k$ is called the auxiliary equation and  $y=\mathrm{e}^{kt}$ is called a trial solution}

\item  Find the auxiliary equation for the following ODEs and find the solutions $k$ for these auxiliary equations. Note if there are one or two solutions and if they are real or complex.
\begin{enumerate}
\item $$  \frac{\diff^2 y}{\diff t^2}  -9 y = 0 $$ 
\item $$  \frac{\diff^2 y}{\diff t^2}+  \frac{\diff y}{\diff t} -2 y = 0 $$ 
\item $$  \frac{\diff^2 y}{\diff t^2}+  2\frac{\diff y}{\diff t} + y = 0 $$ 
\item $$  \frac{\diff^2 y}{\diff t^2}  +9 y = 0 $$ 
\end{enumerate}

\item Considering 4(d) above the auxiliary equation tells us the solutions are $\mathrm{e}^{3\mathrm{i}t}$ and $\mathrm{e}^{-3\mathrm{i}t}$.  Using the fact that  $\mathrm{e}^{\mathrm{i}\theta}= \cos \theta + \mathrm{i} \sin\theta$ show that 

$$\mathrm{e}^{3\mathrm{i}t} + \mathrm{e}^{-3\mathrm{i}t} = 2 \cos 3t$$

 and

$$\mathrm{e}^{3\mathrm{i}t} - \mathrm{e}^{-3\mathrm{i}t} = 2\mathrm{i} \sin 3t.$$

{\em In general if we have a solution $ y= C \mathrm{e}^{\beta\mathrm{i}t}  +D \mathrm{e}^{-\beta\mathrm{i}t}$ then $y = A \cos \beta t + B \sin \beta t$ where $A= (C+D)$ and $B = (C-D)/\mathrm{i}$  is also a solution.}

\item In 4(c) we only get one solution $\mathrm{e}^{-t}$. Check that in this case $t\mathrm{e}^{-t}$ is also a solution.

{\em In general if the auxiliary equation has complex roots $k=\alpha \pm \mathrm{i}\beta$ then the real solutions are of the form $y(t)= \mathrm{e}^{\alpha t}\left( A\cos \beta t + B \sin \beta t\right)$ }

\item 
\begin{enumerate}
\item Check that the auxiliary equation of
$\frac{\diff^2 y}{\diff t^2}+ 2 \frac{\diff y}{\diff t} +4 y = 0 $
has roots $k=-1 \pm \sqrt{3}\mathrm{i}$.
\item The general solution is going to be $y(t)= \mathrm{e}^{- t}\left( A\cos \sqrt{3} t + B \sin \sqrt{3} t\right)$ check that the special case $y(t)= \mathrm{e}^{- t}\cos \sqrt{3} t $ {\em is} actually a solution.
\end{enumerate}








\end{enumerate}
\vfill\eject
\end{document}
