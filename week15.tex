\documentclass[11pt,a4paper]{article}

\usepackage[twoside,a4paper,hmarginratio=3:2,vmarginratio=1:1,bmargin=2.54cm]{
geometry}                % See geometry.pdf to learn the layout options. There are lots.
\geometry{a4paper}                   % ... or a4paper or a5paper or ... 
%\geometry{landscape}                % Activate for for rotated page geometry
%\usepackage[parfill]{parskip}    % Activate to begin paragraphs with an empty line rather than an inden

\usepackage{graphicx}
\usepackage{graphics}
\usepackage{amssymb}
\usepackage{amsmath}
\usepackage{epstopdf}
\usepackage{longtable}
\usepackage{lscape}
%\usepackage{savepapr}
\usepackage{fancyhdr}
\usepackage{fancybox}
\usepackage{indentfirst}
\usepackage{ifthen}
\usepackage{comment}
\usepackage{flafter}
%%%\usepackage{reenumi}
\usepackage{bibentry}
%\usepackage{hyperref}

\usepackage[latin1]{inputenc}
\usepackage{amsmath}
\usepackage{amsfonts}
%\usepackage{makeidx}
\usepackage{bm}
\usepackage{multicol}
\usepackage{color}

\newcommand{\dif}{\mathsf{d}}

%%---------------------------------------------------------------  DJS Definitions 
\def\ffrac#1#2{\leavevmode\kern.1em
\raise.5ex\hbox{\the\scriptfont0 #1}\kern-.1em
/\kern-.15em\lower.25ex\hbox{\the\scriptfont0 #2}}
\def\half{\frac{1}{2}}
\def\hhalf{\ffrac{1}{2}}
\def\bA{\mathbf{A}}
\def\bB{\mathbf{B}}
\def\bC{\mathbf{C}}
\def\bD{\mathbf{D}}
\def\bI{\mathbf{I}}
\def\bP{\mathbf{P}}
\def\bX{\mathbf{X}}
\def\tough{$\!\!\!{}^\star\>$}
\newcommand{\R}{{\mathbb{R}}}


%% Change this boolean to true to compile solutions.
\newboolean{mynotes}
\setboolean{mynotes}{true}

%%---------------------------------------------------------------------------------------
\begin{document}

\begin{center} 
{\bf Maths Problems for 2013--14 | CHEN10072 \\
\ifthenelse{\boolean{mynotes}}{(With solutions) \\}%
{}%
Bill Lionheart\\
 January 2014}
\end{center}
\section*{Week 15}
\begin{enumerate}
 

\item\label{q1171}  If
 \[\bA \,= \left( \begin{array}{rr} 1&2\\-2&4 \end{array} \right),\] calculate
$\bA^{-1}$.   By direct calculation show that
 $\bA\bA^{-1} =\bA^{-1}\bA=\bI $.

\ifthenelse{\boolean{mynotes}}{
\noindent\hrulefill
{\bf Solution}
\[\det \bA= 1\times 4 - 2\times -2=8 \]
\[\bA^{-1}=\left(
\begin{array}{cc}
 \frac{1}{2} & -\frac{1}{4} \\
 \frac{1}{4} & \frac{1}{8}
\end{array}
\right)
\]
\noindent\hrulefill
}{}%


\item\label{q1172} Consider the linear equations
\begin{eqnarray*}
3x\,+\,2y&=&1\\ 
-2x\,+\,5y&=&-7.
\end{eqnarray*}
	\begin{enumerate}
	\item Write this equations in matrix form, i.e. $\bA \mathbf{x}=\mathbf{b}$.
	\item Calculate $\bA^{-1}$. 
	\item Use $\bA^{-1}$ to obtain the values of $x$ and $y$.
	\item Check your working by showing your answer fits the original equations.
	\end{enumerate}

\ifthenelse{\boolean{mynotes}}{
\noindent\hrulefill
{\bf Solution}

\[ \left(
\begin{array}{rr}
 3 & 2 \\
 -2 & 5
\end{array}
\right)^{-1}= \left(
\begin{array}{rr}
 \frac{5}{19} & -\frac{2}{19}
   \\
 \frac{2}{19} & \frac{3}{19}
\end{array}
\right)\]
\[  \left(
\begin{array}{rr}
 \frac{5}{19} & -\frac{2}{19}
   \\
 \frac{2}{19} & \frac{3}{19}
\end{array}
\right) 
\left(
\begin{array}{r}
1 \\ -7
\end{array}
\right) = 
\left(
\begin{array}{r}
1 \\ -1
\end{array}
\right) 
\]
\begin{eqnarray*}
3\times 1\,+\,2\times -1&=&1\\ 
-2\times 1 \,+\,5\times -1 &=&-7.
\end{eqnarray*}

\noindent\hrulefill
}{}%

\item\label{qdjsp2}
\begin{enumerate}
\item
Evaluate the following  $2\times 2$  determinants:
 $$ d_1= \left| \begin{array}{rr} 1&2 \\ -2 & 5 \end{array} \right| , \qquad
	d_2=  \left| \begin{array}{rr}  \ffrac{5}{9} &- \ffrac{2}{9}  \\  \ffrac{2}{9}  &  \ffrac{1}{9} \end{array} \right|. $$ 
How are  $d_1$ and $d_2$ related and why?
\item\tough
Next, evaluate the following  $3\times 3$ determinants:
 $$  d_3 = \left| \begin{array}{rrr} 1&-2&1 \\ 2 & 3 & -1\\-1&-2&2 \end{array} \right|, \qquad
	d_4=  \left| \begin{array}{rrr}2 & 3 & -1\\  1&-2&1 \\ -1&-2&2 \end{array} \right|. $$ 
How are  $d_3$ and $d_4$  related and why? \emph{Hint:  Use the properties of permutation matrices.}
\end{enumerate}
\ifthenelse{\boolean{mynotes}}{
\noindent\hrulefill

{\bf Solution}

\begin{enumerate}
\item $d_1 = 1\times 5- 2 \times -2 =9$ whereas $d_2 = \frac{5}{9}\times\frac{1}{9}-\frac{2}{9}\times \frac{-2}{9}=\frac{1}{9}$ 

The determinant of the inverse of a matrix the reciprocal of the determinant of that matrix.
\item $d_3=9$, $d_4=-9$. Swapping the first two rows corresponds to multiplying on the left by the permutation matrix $\bP_{12}$ from the previous week's problems. As $\det(\bP_{12})=-1$ the determinant changes sign. 
\end{enumerate}
\noindent\hrulefill
}{}%



\item\label{q1205}\tough
Suppose that you have square matrices, $\bA, \bB$ and $\bC$, with the properties
$\bA\bB=\bI$ and $\bC\bA =\bI$, where $\bI$ is the identity matrix.
Use this information to prove that $\bC = \bB$.

You may assume that  $\bD \bI = \bI\bD = \bD$  for any square matrix $\bD$. You may
not assume that $\bA \bA^{-1} = \bA^{-1} \bA = \bI$. 
The  point of the question is to establish that the
left-hand and right-hand inverses are identical and the above  statement assumes this.
\emph{Hint: Do not use  inverse matrices in this question. 
Also do not look at matrix components --- consider
matrices as single objects and make use of their multiplication properties. 
The proof is just a few lines long!}


\ifthenelse{\boolean{mynotes}}{
\noindent\hrulefill

{\bf Solution}

We have $\bA\bB=\bI$ and multiplying on the left by $\bC$ gives $\bC\bA\bB=\bC\bI$. Using 
$\bC\bA =\bI$ we have $\bI\bB=\bC\bI$. We then use the given property of the identity matrix $\bD \bI = \bI\bD = \bD$  for any square matrix $\bD$ to conclude that $\bB = \bC$.

\noindent\hrulefill

}{}%

\end{enumerate}

\vfill\eject




\end{document}
