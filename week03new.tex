\documentclass[11pt,a4paper]{article}

\usepackage[twoside,a4paper,hmarginratio=3:2,vmarginratio=1:1,bmargin=2.54cm]{
geometry}                % See geometry.pdf to learn the layout options. There are lots.
\geometry{a4paper}                   % ... or a4paper or a5paper or ... 
%\geometry{landscape}                % Activate for for rotated page geometry
%\usepackage[parfill]{parskip}    % Activate to begin paragraphs with an empty line rather than an inden

\usepackage{graphicx}
\usepackage{graphics}
\usepackage{amssymb}
\usepackage{amsmath}
\usepackage{epstopdf}
\usepackage{longtable}
\usepackage{lscape}
%\usepackage{savepapr}
\usepackage{fancyhdr}
\usepackage{fancybox}
\usepackage{indentfirst}
\usepackage{ifthen}
\usepackage{comment}
\usepackage{flafter}
%%%\usepackage{reenumi}
\usepackage{bibentry}
%\usepackage{hyperref}

\usepackage[latin1]{inputenc}
\usepackage{amsmath}
\usepackage{amsfonts}
%\usepackage{makeidx}
\usepackage{bm}
\usepackage{multicol}
\usepackage{color}

\newcommand{\dif}{\mathsf{d}}

%%---------------------------------------------------------------  DJS Definitions 
\def\ffrac#1#2{\leavevmode\kern.1em
\raise.5ex\hbox{\the\scriptfont0 #1}\kern-.1em
/\kern-.15em\lower.25ex\hbox{\the\scriptfont0 #2}}
\def\half{\frac{1}{2}}
\def\hhalf{\ffrac{1}{2}}
\def\bA{\mathbf{A}}
\def\bB{\mathbf{B}}
\def\bC{\mathbf{C}}
\def\bD{\mathbf{D}}
\def\bI{\mathbf{I}}
\def\bP{\mathbf{P}}
\def\bX{\mathbf{X}}
\def\tough{$\!\!\!{}^\star\>$}
\newcommand{\R}{{\mathbb{R}}}


%% Change this boolean to true to compile solutions.
\newboolean{mynotes}
\setboolean{mynotes}{true}

\begin{document}
	
\begin{center} 
{\bf Maths Problems for  CHEN10072 \\
\ifthenelse{\boolean{mynotes}}{(With solutions) \\}

 Bill Lionheart
}
\end{center}
\hrule
\smallskip


\section*{Week 3}

\smallskip


\begin{enumerate}
\item\label{qdjsn1} 
Verify that the nonlinear equation 
$${x}^{1/2} \sin (x) = 1$$ has at
least one solution in the interval (1.1,1.3). Hence find a solution  
which is accurate to two decimal places using
\begin{enumerate}
%\renewcommand{\theenumii}{\roman{enumii}}
\item an interval halving method,
\item a Newton iteration method (with initial guess $x_0 = 1.2$),
\end{enumerate}

\ifthenelse{\boolean{mynotes}}{
\noindent\hrulefill
{\bf Solution}
\noindent\hrulefill
\begin{enumerate}
\item
We define $f(x)={x}^{1/2} \sin (x) - 1$.  Evaluating  the function at
the ends of the interval we  find that
$f(1.1)= -0.0653<0$ and $f(1.3)= 0.0986>0$. Thus,  since the
function changes sign in $[1.1,1.3]$ we know that  a root lies inside this interval.
Next,  computing the midpoint $x=1.2$ we
find that $f(1.2)=  0.0210>0$  which mans that a root lies in the
smaller interval $[1.1,1.2]$ . Repeating the process six times gives the following 
sequence: 
\begin{center}
\begin{tabular}{| ccc | ccc | }\hline
%{\tt its} &\multicolumn{2}{c|}{{\tt 5}} & \multicolumn{2}{c|}{{\tt 10}} & \multicolumn{2}{c|}{{\tt 20}} \\
$x_a$ & $x_m= (x_a+x_b)/2$ & $x_b$ & $f(x_a)$ & $f(x_m)$ & $f(x_b)$  \\ \hline
    1.1000 &1.1500 & 1.2000 &  $<0$  &  $<0$  &  $>0$ \\ 
    1.1500 & 1.1750 & 1.2000 &  $<0$  &  $>0$  &  $>0$ \\ 
        1.1500 & 1.1625 & 1.1750  &  $<0$  &  $<0$  &  $>0$ \\ 
    1.1625 & 1.1687  & 1.1750 &  $<0$  &  $<0$  &  $>0$ \\ 
    1.1687 & 1.1718  & 1.1750  &  $<0$  &  $<0$  &  $>0$ \\ 
    1.1718 & 1.1734  & 1.1750 &  $<0$  &  $<0$  &  $>0$ \\  
\hline
\end{tabular}
\end{center}
We conclude that the root is ${\bf 1.17...}$.
\item
Differentiating $f(x)$ we have that $f'(x) = {x}^{1/2} \cos (x) +  {x}^{-1/2} \sin(x)/ 2$.
Given the starting value $x_0=1.2$,  Newton's iteration is to compute the
sequence $x_1, x_2, \ldots$ using the formula
$$
x_{n+1} = x_n - {f(x_n)\over f'(x_n)}, \qquad n=0,1,2,\ldots
$$
A direct computation gives the following results:

\begin{center}
\begin{tabular}{| c |  rr | }\hline
%{\tt its} &\multicolumn{2}{c|}{{\tt 5}} & \multicolumn{2}{c|}{{\tt 10}} & \multicolumn{2}{c|}{{\tt 20}} \\
$x_n$ & $f(x_n)$ & $f'(x_n)$  \\ \hline
    1.20000 & 2.099e-02 & 8.223e-01\\
   1.17447 & -2.781e-04 & 8.439e-01\\
   1.17480 & -4.480e-08 & 8.436e-01\\
   1.17480 & -1.332e-15 & 8.436e-01 \\
 \hline
\end{tabular}
\end{center}
We conclude that the root is ${\bf 1.17480...}$.

\end{enumerate}
}{}%



\item\label{qdjsn2}
Derive a Newton method to calculate the fifth root of a given real 
number, $s$ say. Use your method to calculate $5^{1/5}$ correct to 
four decimal places.

\ifthenelse{\boolean{mynotes}}{
\noindent\hrulefill
{\bf Solution}
\noindent\hrulefill

We introduce the function $f(x)= x^5 -s$. Note that $f(x)=0$ when $x=s^{1/5}$.
Given a starting value $x_0=1$, say,  Newton's iteration is to compute the
sequence $x_1, x_2, \ldots$ using the formula
$$
x_{n+1} = x_n - {f(x_n)\over f'(x_n)} = x_n -  {x_n^5 -s \over 5x_n^4}, \qquad n=0,1,2,\ldots
$$
A direct computation with $s$ set to $5$ gives the following results:

\begin{center}
\begin{tabular}{| c |  rr | }\hline
%{\tt its} &\multicolumn{2}{c|}{{\tt 5}} & \multicolumn{2}{c|}{{\tt 10}} & \multicolumn{2}{c|}{{\tt 20}} \\
$x_n$ & $f(x_n)$ & $f'(x_n)$  \\ \hline
1.00000 & -4.000e0 & 5.000e0 \\
1.80000 & 1.389e1 & 5.248e1 \\
1.53526 & 3.529e0 &2.777e1 \\
1.40821 & 5.377e-01 &1.966e1 \\
1.38086 & 2.048e-02 &1.817e1 \\
1.37973 & 3.341e-05 &1.811e1 \\
1.37973 & 8.930e-11 & 1.811e1 \\
 \hline
\end{tabular}
\end{center}
We conclude that  $5^{1/5}$  is ${\bf 1.3797...}$.




}{}%

\item We aim to solve the simultaneous equations 
\begin{equation*}
x^3 +y =1, \, y^3 -x =-1
\end{equation*}
\begin{enumerate}
	\item Sketch the two curves and convince your self they intersect only once at $(x,y)=(1,0)$
	\item Calculate the Jacobian matrix $\mathbf{J}$ (ie matrix  of partial derivatives) of this system
	\item Using the iteration 
	$$ \mathbf{x}_{n+1}=\mathbf{x}_{n} - \mathbf{J}^{-1}\mathbf{f}(\mathbf{x}_n) $$ with $\mathbf{x}_{0}=(0.5\,  0.5)^T$  do one iteration of Newton's method. Is it closer to the solution?
	\item Using Matlab, Octave or software of your choice find out how many iterations are needed to get the solution to three decimal places.
\end{enumerate}
\ifthenelse{\boolean{mynotes}}{
	\noindent\hrulefill
	{\bf Solution}
	\noindent\hrulefill

\begin{enumerate}
	\item A plot of the graph looks like this. Seems to have a solution near $x=0, y=1$ (well that is actually the solution!)
	
	 \includegraphics[scale=0.3]{Week03Q3curves}
	\item $$\mathbf{J}=\left(
	\begin{array}{cc}
		3 x^2 & 1 \\
		-1 & 3 y^2 \\
	\end{array}
	\right)$$
	and $\det \mathbf{J}=1+8x^2 y^2$ so
	\[
	\mathbf{J}^{-1} = \left(
	\begin{array}{cc}
	\frac{3 y^2}{9 x^2 y^2+1} & -\frac{1}{9 x^2
		y^2+1} \\
	\frac{1}{9 x^2 y^2+1} & \frac{3 x^2}{9 x^2
		y^2+1} \\
	\end{array}
	\right)
	\]
	\item 
	
	The first iteration gets us to $x=-0.2,y= 1.4$  so yes it is a bit closer!
	
	\item subsequent iterations give us
	
\begin{center}
	\begin{tabular}{| c |  ll | }\hline
		
		$n$& $x_n$ & $y_n$  \\ \hline
	1&	$-0.411632$& 1.0334 \\
	2&	$	-0.171321$& 0.947592\\
	3&	$	-0.0283247$&0.992437\\
	4&	$		-0.000302864$& 0.999955\\
	5 & $-6.16617\times 10^{-9}$& 1.0 \\
	\\
		\hline
	\end{tabular}
\end{center}

	
\end{enumerate}	
}{}

\end{enumerate}
\hfill {\tiny Last modified  \today}
\vfill\eject
\end{document}
