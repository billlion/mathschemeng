\documentclass[12pt]{article}
%\usepackage{mathexam,amssymb,theorem}
\usepackage{CEASexam,amsfonts}
\usepackage {amsmath,amssymb}


\coursename{Engineering Mathematics 2}
\examinationdate{MOCK End of Semester  Examination}
\examinationtime{ xxx -- yyy }

\coursecode{CHEN10072}
\duration{One and a Half  Hours}
\rubric{Answer {\bf all} parts of  {\bf Section A}
(40 marks in all)
        and\\
        {\bf one} of the two questions in {\bf Section B}
(10 marks)}

\calculators{Electronic calculators may be used, provided that
they CANNOT store text.}
% The total number of marks on the paper is 50.  This represents 80%
% of the MATH36041 mark and 55% of the MATH66041 mark


%%%%%%%%%%%%%%%%%%%%%%%%%%%%%%%%%%
% Make uppercase Greek characters italic.
% Copied from latex.ltx and changed second digit from 0 (roman font)
% to 1 (math italic).
\mathchardef\Gamma="7100
\mathchardef\Delta="7101
\mathchardef\Theta="7102
\mathchardef\Lambda="7103
\mathchardef\Xi="7104
\mathchardef\Pi="7105
\mathchardef\Sigma="7106
\mathchardef\Upsilon="7107
\mathchardef\Phi="7108
\mathchardef\Psi="7109
\mathchardef\Omega="710A
%%%%%%%%%%%%%%%%%%%%%%%%%%%%%%%%%
% Define a few macros specific to this exam and this particular typist.
\def\ffrac#1#2{\leavevmode\kern.1em
\raise.5ex\hbox{\the\scriptfont0 #1}\kern-.1em
/\kern-.15em\lower.25ex\hbox{\the\scriptfont0 #2}}
\def\half{\frac{1}{2}}
\def\hhalf{\ffrac{1}{2}}
\def\bA{\mathbf{A}}
\def\bB{\mathbf{B}}
\def\bC{\mathbf{C}}
\def\bD{\mathbf{D}}
\def\bI{\mathbf{I}}
\def\bP{\mathbf{P}}
\def\bX{\mathbf{X}}
\def\tough{$\!\!\!{}^\star\>$}
\newcommand{\R}{{\mathbb{R}}}
\newcommand{\dif}{\mathsf{d}}

\begin{document}
\maketitle

\null\bigskip
{\bf General Instruction.} All numerical working must be shown.


\section
\sectionheader{Answer \jazzy{all} questions}

\question
Given \[ \bA \,=\, \left( \begin{array}{rr} 1 &2 \\ 0 & 4 \end{array}\right) \]
calculate $\det\,\bA$ and $\bA^{-1}$ 
\marks{2}

\question
 Let
\[ \bA \,=\, \left( \begin{array}{rr} 1 &2 \\ -2 &2 \end{array} \right) \quad
\hbox{and} \quad
\bB \,=\, \left( \begin{array}{rr} -3 &1 \\ -1 & 0 \end{array} \right).\]

Compute the following matrices:
	\begin{enumerate}
	\item $\bA+ 2\bB$.  Is it true that $ \bA + 2\bB= 2\bB + \bA$?
	\item $\bA^T+  \bB^T$. Is it true that $( \bA + \bB)^T = \bA^T + \bB^T$?
	\item $\bA \bB$ and $\bB \bA$.    Is it true that $ \bA \bB= \bB \bA$?
	\item $\bB^T \bA^T$.  Is it true that $(\bA \bB)^T= \bB^T \bA^T$?
	\marks{8}
	\end{enumerate}



\question
Evaluate the following  $3\times 3$ determinants:
 $$  d_1 = \left| \begin{array}{rrr} 1&-2&1 \\ 2 & 3 & -1\\-1&-2&2 \end{array} \right|, \qquad
	d_2=  \left| \begin{array}{rrr}1 & 1 & -2\\  2& -1 &3 \\ -1&2&-2 \end{array} \right|. $$
How are  $d_1$ and $d_2$  related and why?

\noindent
 [Hint:   use the fact that
$\hbox{det}(\bA \bB) =  \hbox{det}(\bA) \times \hbox{det}(\bB)$.]
\marks{4}

\question
Given
\vspace{-0.5cm}
\begin{eqnarray*}
x\,+\,y\,+\,z&=&3\\
x\,-\,y\,+\,2z&=&2\\
y\,-\,z&=&5 .
\end{eqnarray*}
            \begin{enumerate}
            \item Write the system of  equations in the form of $\bA \mathbf{x}=\mathbf{b}$.
            \item\label{q1207b}  Calculate the determinant of {\bf A}.
            \item  Solve for $x,y,z$ using Gaussian elimination without pivoting.\marks{6}
    \end{enumerate}
	

\question
Solve analytically the following differential equation problem: find $y(x)$ such that
$$ \frac{\dif y}{\dif x}\,=\,  x\sqrt{1\,-\,y^2}   ; \qquad y=0\>\hbox{ when }\> x=0. $$
\noindent
 [Hint:   use the fact that
$\displaystyle \int {1 \over \sqrt{1-x^2}} \, dx=  \sin^{-1} x$.]
\marks{4}

\question
Solve the following problem by finding an  integrating factor:  find $y(x)$ such that
$$  \frac{\dif y}{\dif x}\,+\,\frac{y}{x} = \sin{x}   ; \qquad y=0\>\hbox{ when }\> x=\pi. $$
\marks{5}

\question
Compute a numerical solution to the differential equation problem
$$ \frac{\dif y}{\dif t} =  -y + t +1   ; \qquad y=1\>\hbox{ when }\> x=0, $$
by taking four steps of the forward Euler method with
step size $h=0.1$.
\marks{5}

\question
Write down  Newton's iteration   for finding the roots of
a function $f(x)$. Hence or otherwise,
determine an iterative method to calculate the cube root of a given real
number, $\beta$ say. Starting with an initial guess of $x_0 =2$, run the
iteration and   calculate $\root 3 \of 10$ correct to  four decimal places.
\marks{6}

%%%%%%%%%%%%%%%%%%%%%%%%%%%%% SECTION B
\section
\sectionheader{Answer \jazzy{one} of the two questions}

\question
Let $\bA$ represent a general $k\times k$ matrix.
\begin{enumerate}
\item
Explain as succinctly as possible how you might use  Gaussian Elimination to find the inverse of
the matrix $\bA$.  \marks{4}

\item
Apply  the  technique to compute   the inverse of  the following matrix
$$ \bA = \left( \begin{array}{cc} -2 &2  \\ 1 & 2 \end{array} \right). $$
\null\vspace{-0.6cm}
 \marks{3}

 \item
Describe the process of    Gaussian Elimination  {\it with partial pivoting} and give the main reason
why including pivoting is  important when solving linear equation systems using
a computer.
\null\vspace{-0.4cm} \marks{3}
\end{enumerate}
\bigskip
\question
Explain what is meant by  the terminology of a {\it nonlinear  first-order} differential equation
problem
$$
{\dif y\over \dif t} = f(t,y)
$$
and give an example of such an equation.
\null\vspace{-0.4cm} \marks{2}

\begin{enumerate}
\item
Given some initial condition and a  step length of $h$,
write down the general update formula for computing the {\it implicit} Euler
solution  $y^*_{n+1}$  from the previous estimate $y^*_n \approx y(nh)$.
Explain carefully how you would  compute  $y^*_{n+1}$ in practice.
Give one reason why  the  {\it implicit} Euler method might be used
in preference to the  {\it forward} Euler method, when solving a nonlinear
differential equation problem.
\null\vspace{-0.2cm} \marks{5}

\item
 Take a  step size  of $h=0.1$  and
compute the first two steps of the  {\it implicit} Euler  method
applied to the linear problem
$$ \frac{\dif y}{\dif t} =  -y + t +1   ; \qquad y=1\>\hbox{ when }\> x=0. $$
\marks{3}
\end{enumerate}


\finished

\end{document}
